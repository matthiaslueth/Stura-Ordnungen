%\addchap[AE-Ordnung]{AE-Ordnung\\der Studentenschaft der TU Dresden zu Finanzordnung § 40 Abs. 2}
\markright{Ordnung zur Nutzung der Fachschaftenverteiler}
\setcounter{section}{0}
\begin{multicols}{2}


\section{Zweck der Ordnung}

\Abs \Satz Zweck der Ordnung ist die Regulierung des Mailverkehrs über die Verteiler der Fachschaften\. Insbesondere ist geregelt, wer Zugriff hat und welche Inhalte über die Verteiler versendet werden dürfen.


\section{Zugriffsberechtigung}

\Abs \Satz Jeder Fachschaftsrat bestimmt eigenständig seine Verantwortlichen für den Verteiler und meldet sie dem Studentenrat auf dem entsprechenden Bogen\. Die Verantwortlichen bekommen vom Studentenrat funktionsbezogene E-Mail-Adressen zugeordnet und haben von dieser Adresse aus Sendeberechtigung für ihren Verteiler\. Der Versand von E-Mails ist ferner ausschließlich auf Beschluss des Fachschaftsrates gestattet.


\section{Verfahrensweise}

\Abs \Satz Das ZIH generiert Mailing-Listen, die die Studierenden des entsprechenden Fachbereichs umfassen\. Die Verantwortlichen gemäß § 2 dürfen mit ihrer funktionsbezogenen E-Mail-Adresse die Studierenden der Mailing-Listen unter Wahrung der unter §4 aufgeführten Inhalte anschreiben\. Eine direkte Weitergabe von personenbezogenen Daten (speziell der E-Mail-Adressen) seitens des ZIH erfolgt nicht.

\section{Zugelassene Inhalte}

\Abs \Satz Es sind nur Inhalte zugelassen, die der Erfüllung der Aufgaben der Studentenschaft dienen\.
Die Aufgaben der Studentenschaft sind gemäß § 24 Abs. 3 SächsHSFG die
\begin{enumerate}
\item Wahrnehmung der hochschulinternen, hochschulpolitischen, sozialen und kulturellen Belange der Studenten,
\item Mitwirkung an Evaluations- und Bewertungsverfahren gemäß § 9 Abs. 2 und 3 SächsHSFG,
\item Unterstützung der wirtschaftlichen und sozialen Selbsthilfe der Studenten,
\item Unterstützung der Studenten im Studium,
\item Förderung des Studentensports unbeschadet der Zuständigkeit der Hochschule,
\item Pflege der regionalen, überregionalen und internationalen Studentenbeziehungen und die Förderung der studentischen Mobilität,
\item Förderung der politischen Bildung und des staatsbürgerlichen Verantwortungsbewusstseins der Studenten.
\end{enumerate}
Jede E-Mail, die über die Fachschaftsverteiler gesendet wird, muss folgende Anmerkung als Fußtext enthalten:
Diese E-Mail wurde gemäß der Ordnung zur Nutzung der Fachschaftsverteiler der Studentenschaft der Technischen Universität Dresden versendet. Ordnungsverstöße sind der Geschäftsführung des Studentenrates [Kontakt: gf@stura.tu-dresden.de] anzuzeigen. Beschwerden müssen an den Absender sowie als Kopie an den Geschäftsführer für Lehre und Studium [Kontakt: lust@stura.tu-dresden.de] gesendet werden.

\section{Verhaltensregeln}

\Abs \Satz Die Fachschaftsverteiler sollen in vernünftigem Maße genutzt werden\. Nachrichten sollten daher im Regelfall nicht häufiger als wöchentlich versendet werden\. Zwecks dessen sollen die anfallenden Nachrichten gesammelt und zusammengefasst versendet werden\. Werbung für Veranstaltungen ist generell nur zulässig, wenn sie der Erfüllung der Aufgaben der Studentenschaft gemäß § 4 dient\. Wissenschaftliche Umfragen können in einer E-Mail, die nicht ausschließlich dem Versand der Umfrage dient, nach dem Fußtext angefügt oder in gesammelter Form maximal einmal wöchentlich versendet werden.

\section{Ordnungsverstöße und Beschwerden}

\Abs \Satz Es gelten die entsprechenden Bestimmungen der am Versand beteiligten Einheiten der Technischen Universität Dresden\.
Bei Ordnungsverstößen oder berechtigten Beschwerden der E-Mail-Empfänger/innen behält sich die Geschäftsführung des Studentenrates vor, den Verteiler der entsprechenden Fachschaft zu sperren, bis die Beschwerde bearbeitet ist und die notwendigen Konsequenzen gezogen wurden\.
Bei wiederholtem Ordnungsverstoß oder wiederholter Beschwerde kann die Geschäftsführung den Verteiler der entsprechenden Fachschaft bis zum Ende der Amtsperiode sperren.

%Ende für multicols
\end{multicols}

\nopagebreak
\vspace{1cm}

\footnotesize
Inkraftgetreten am 13. November 2014.\\


\normalsize
~\\*[2cm]
\begin{center}
\hspace*{\fill}
\parbox{7cm}{Robert Georges\\GF Finanzen}
\hfill\parbox{7cm}{Jan-Malte Jacobsen \\GF Hochschulpolitik}
\hspace*{\fill}
\end{center}
