%\addchap[AE-Ordnung]{AE-Ordnung\\der Studentenschaft der TU Dresden zu Finanzordnung § 40 Abs. 2}
\markright{AE-Ordnung}
\setcounter{section}{0}
\begin{multicols}{2}


\section{Allgemeines}

\Abs \Satz Gemäß §40 der Finanzordnung werden im Folgenden die Grundzüge der Art und Weise der Zahlungen von Aufwandsentschädigungen (AE) geregelt.

\Abs \Satz Als Anspruchszeitraum gilt genau ein Kalendermonat. Für die Sportobleute gilt als Anspruchszeitraum ein Semester.


\section{AE-Berechtigte}

\Abs \Satz AEs können beantragt werden durch
\begin{enumerate}
\item Referatsmitarbeiterinnen,
\item Referentinnen,
\item Geschäftsführerinnen,
\item Sportobleute,
\item Mitarbeiterinnen von Projekten des StuRa,
\item Ausschussmitarbeiterinnen, falls dies bei der Einrichtung des Ausschusses so geregelt wurde,
\item Mitglieder des Sitzungsvorstandes.
\end{enumerate}


\section{AE-Beantragung}

\Abs \Satz Anträge auf Aufwandsentschädigung müssen spätestens am 10. Tag nach dem Ende des Anspruchszeitraums gestellt werden.

\Abs \Satz Anträge auf Aufwandsentschädigung müssen begründet werden.

\Abs \Satz Die beantragten Aufwandsentschädigungen sind so aufzuschlüsseln, dass sie den jeweiligen Sachkonten des Wirtschaftsplanes zugeordnet werden können.

\section{Festlegung der AE Höhe}

\Abs \Satz Für die nach §2 (2) definierten Ämter können von Referatsmitarbeiterinnen 70 Euro, von Referentinnen 125 Euro und von Geschäftsführerinnen 210 Euro als AE beantragt werden.

\Abs \Satz Bei unvorhergesehenen und außerordentlichen Aufgaben oder Mitarbeit an Projekten kann über die in (1) genannte Summe bis zu 350 Euro beantragt werden.

\Abs \Satz  Die studentischen Sportobleute des Universitätssportzentrums der TU Dresden können eine AE in Höhe von maximal 200 Euro pro Person und Semester erhalten\. Mitglieder des Sitzungsvorstandes werden wie Referentinnen behandelt.

\Abs \Satz Die Höhe der Aufwandsentschädigung, die vom StuRa gezahlt wird, ist auf 350 Euro pro Person und Monat begrenzt. 

\section{Beschlussfassung über AE Anträge}

\Abs \Satz Die Beschlussfassung über Aufwandsentschädigungen wird in nichtöffentlicher Sitzung befunden.

\Abs \Satz Die Anträge auf Aufwandsentschädigung sowie deren Begründungen müssen allen StuRa- Mitgliedern zugänglich gemacht werden. Näheres wird in der Durchführungsbestimmung geregelt.

\Abs \Satz Die Aufwandsentschädigungen der Geschäftsführerinnen werden vom StuRa-Plenum beschlossen.

\Abs \Satz Sonstige Aufwandsentschädigungen werden von der Geschäftsführung beschlossen.


\section{Sonstige und Schlussbestimmungen}

\Abs \Satz Diese Ordnung gilt ab dem nächsten Anspruchszeitraum (§1, Absatz 2) nach Erlass.

%Ende für multicols
\end{multicols}

\nopagebreak
\vspace{1cm}

\footnotesize
Inkraftgetreten am 30. August 2012.\\


\normalsize
~\\*[4cm]
\begin{center}
\hspace*{\fill}
\parbox{7cm}{Robert Georges\\GF Finanzen}
\hfill\parbox{7cm}{Jan-Malte Jacobsen \\GF Hochschulpolitik}
\hspace*{\fill}
\end{center}
