%Header für spaltenlose Version
%\addchap[Beitragsordnung der Studentenschaft der TU Dresden]{Beitragsordnung\\der %Studentenschaft der TU Dresden}
%\markright{Beitragsordnung}
%\setcounter{section}{0}


%Header für multicols
\markright{Beitragsordnung}
\setcounter{section}{0}
\begin{multicols}{2}
 

\section{Beitragszweck}

\Abs \Satz  Die Studentenschaft der TU Dresden erhebt zur Durchführung ihrer Aufgaben von ihren Mitgliedern Beiträge [§~2~Abs.~2 Grundordnung der Studentenschaft der TU Dresden]. 

\section{Beitragshöhe}

\Abs \Satz  Der Beitrag ist in folgender Höhe für folgende Zwecke bestimmt:
\begin{enumerate}
\item Für den StuRa 3,70~Euro
\item Für die Fachschaften 0,90~Euro
\item Für das Studentenjahresticket VVO und SPNV Sachsen 351,60~Euro pro Studienjahr (Wintersemester und nachfolgendes Sommersemester)
\end{enumerate}

\Abs \Satz  Studentinnen, die erstmals im Sommersemester immatrikuliert werden, zahlen für den verbleibenden Gültigkeitszeitraum nur den halben Beitrag des Studentenjahrestickets. 


\section{Beitragspflicht}

\Abs \Satz  Der Beitragspflicht unterliegen alle Studentinnen, die Mitglied der Studentenschaft der TU Dresden sind.

\Abs \Satz  Fernstudentinnen, Studentinnen, die an Außenstellen der TU Dresden außerhalb des Verbundgebietes des Verkehrsverbundes Oberelbe (VVO) immatrikuliert sind und dort studieren, sowie Studentinnen, die vom Studium beurlaubt sind, sind, sofern sie den Antrag auf Beurlaubung bis zum Ende der Rückmeldefrist gemäß § 6 Abs. 1 Immatrikulationsordnung gestellt haben, während dieser Zeiten von der Zahlungspflicht für die jeweilige Rate gemäß § 5 Abs. 2 für das Studentenjahresticket befreit.


\section{Rückerstattung und Nachkauf}

\Abs \Satz  Der Studentenschaftsbeitrag kann in sozialen Härtefällen aus Mitteln des Studentenrates zurückerstattet werden\. Näheres regelt die Härtefallordnung. 

\Abs \Satz  In folgenden Fällen können Studentinnen auf schriftlichen Antrag an den Studentenrat den Beitragsanteil für das Studentenjahresticket zurück erhalten:
\begin{enumerate}
\item behinderte Studentinnen im Besitz eines Schwerbehindertenausweises mit einem der gültigen Merkzeichen (gem. SGB~IX)
\renewcommand{\labelitemi}{—}
\begin{itemize}
\item aG,
\item Bl,
\item H,
\item G mit gültiger Wertmarke,
\item Gl mit gültiger Wertmarke
\end{itemize}
oder mit anderweitig nachgewiesener Behinderung, die die Nutzung des Studentenjahrestickets verhindert,
\item Ableistung eines Praktikums oder einer sonstigen studienbedingten Anstellung außerhalb des VVO-Verbundgebietes ,
\item Erstellung einer Diplomarbeit bzw. sonstiger Abschlussarbeit studienbedingt außerhalb des VVO-Verbundgebietes,
\item Rücktritt vom Studienplatz,
\item nachträgliche Beurlaubung,
\item Promotion außerhalb des VVO-Verbundgebietes,
\item studienbedingter Auslandsaufenthalt ohne Beurlaubung, 
\item Im- oder Exmatrikulation.
\end{enumerate}

\Abs \Satz  Der Antrag auf Rückerstattung muss spätestens 6 Tage nach Eintreten des Rückerstattungsgrundes beim Studentenrat eingehen, andernfalls kann nur für den Zeitraum nach Antragseingang erstattet werden. 

\Abs \Satz   Als Eingangszeitpunkt eines Antrags auf Erstattung des Beitrags für das Studentenjahresticket gilt der Zeitpunkt, zu dem dieser Antrag und der Studentenausweis dem Studentenrat vorliegen\. Die schriftlichen Unterlagen zum Nachweis der Voraussetzungen für eine Beitragserstattung gemäß §~4~Abs.~2 können binnen sechs Wochen nachgereicht werden. 

\Abs \Satz  Für jeden vollen Monat nach Antragseingang, für den ein Rückerstattungsgrund gemäß § 4 Abs. 2 vorliegt, ist je Monat ein Zwölftel des Jahresticketbeitrags zu erstatten\. Dabei gilt als voller Monat auch der Monat, in dem der Rückerstattungsgrund für maximal 7 Tage nicht vorliegt\. Außer im Fall der Ex- oder Immatrikulation erfolgt keine Rückerstattung von weniger als einem Sechstel des Jahresbeitrags.

\Abs \Satz  Anträge nach Abs. 2 Nr. 1 bis 6, die nach dem 31.8. für das laufende Studienjahr eintreffen, sind abzulehnen\. Bei Verlust des Studentenausweises erfolgt keine Rückerstattung für das jeweilige Semester.

\Abs \Satz  Die Möglichkeit, das Studentenjahresticket nachträglich zu erwerben, haben alle Studentinnen, die nach § 3 von der Beitragspflicht des Studentenjahrestickets befreit sind\. Studentinnen, die nach § 3 Abs. 1 vom gesamten Studentenschaftsbeitrag befreit sind, haben diesen beim Nachkauf des Studentenjahrestickets ebenfalls zeitanteilig nachzuentrichten\. Der Preis für das Studentenjahresticket im Nachkauf beträgt für jeden angefangenen Monat Restgültigkeit ein Zwölftel des Beitragsanteils für das Studentenjahresticket, mindestens jedoch ein Sechstel von diesem.

\Abs \Satz  Studentinnen, die, ohne die Voraussetzungen einer Beitragserstattung gemäß Absatz 2 zu erfüllen, im Laufe eines Studienjahres aus der Studentenschaft der TU Dresden austreten, bleiben bis zum Ende des betreffenden Studienjahres Inhaber des Studentenjahrestickets und verpflichtet, den Beitragsanteil für das Studentenjahresticket zu zahlen.


\section{Beitragserhebung und Fälligkeit}

\Abs \Satz  Der Semesterbeitrag ist in der vom Immatrikulationsamt bekannt gemachten Form einzuzahlen\. Er wird fällig mit der Einschreibung bzw. Rückmeldung.

\Abs \Satz  Der Beitragsanteil für das Studentenjahresticket ist wahlweise mit der Rückmeldung zum Wintersemester eines Studienjahres oder in zwei gleichen Raten zu je 175,80~Euro mit der Rückmeldung zum Wintersemester und zum darauffolgenden Sommersemester einzuzahlen.

\Abs \Satz Sofern mit der Rückmeldung zum Wintersemester nur die erste Rate eingezahlt wurde, entfällt bei Exmatrikulation während oder am Ende des Wintersemesters die Einzahlung der zweiten Rate.

\section{Mittelverwaltung}

\Abs \Satz  Der StuRa zahlt aus der Summe der für ihn gemäß §~2~Abs.~1 bestimmten Mittel jeder Fachschaft einen Sockelbetrag in Höhe von 500,00~Euro.

\Abs \Satz  Der StuRa verwaltet die für ihn bestimmten Mittel entsprechend seiner Finanzordnung\. Die Fachschaften verwalten die ihnen übergebenen Mittel in eigener Verantwortung gemäß der Finanzordnung.

\Abs \Satz  Die Beiträge für das Studentenjahresticket des VVO werden durch das Immatrikulationsamt gemäß der mit diesen Unternehmen getroffenen Vereinbarung direkt überwiesen.

\Abs \Satz  Die Regelungen der §§~3~Abs.~4~und~8~Abs.~2~S.~2 der Finanzordnung bleiben unberührt.

%Ende für multicols
\end{multicols}

\nopagebreak
\vspace{1cm}
Inkraftgetreten am 01.~Juli~2015.
\\


\footnotesize
komplett neu gefasst am 27.~Juni~2013\\

\normalsize
~\\*[4cm]
\begin{center}
\hspace*{\fill}
\parbox{7cm}{Matthias Funke\\GF Finanzen}
\hfill\parbox{7cm}{Andreas Spranger\\GF Hochschulpolitik}
\hspace*{\fill}
\end{center}

