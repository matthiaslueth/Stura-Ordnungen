\addchap[Finanzordnung]{Finanzordnung\\der Studentenschaft der
%TU Dresden}
\markright{Finanzordnung}
\setcounter{section}{0}
\begin{multicols}{2}
 
       
        \begin{description}
            \item[1. Abschnitt] Allgemeines
            \item[2. Abschnitt] Geschäftsführerin Finanzen
            \item[3. Abschnitt] Der Wirtschaftsplan
            \item[4. Abschnitt] Kassenwesen
            \item[5. Abschnitt] Bewilligung von Zahlungen
            \item[6. Abschnitt] Aufwandsentschädigungen
            \item[7. Abschnitt] Prüfungswesen
        \end{description}




\section*{1. Allgemeines}

\section{Übergeordnete Bestimmungen}

\Abs \Satz Für die Haushalts- und Wirtschaftsführung der Studentenschaft sind die Sächsische Haushaltsordnung und das Sächsische Hochschulgesetz maßgebend.

\Abs \Satz Für alle Fälle, in denen diese Ordnung keine Regelungen trifft, sind die in Abs.~1 genannten Bestimmungen anzuwenden.


\section*{2. Geschäftsführerin Finanzen}



\section{Wahl}

\Abs \Satz Ein Mitglied des Studentenrates wird vom StuRa zur Geschäftsführerin Finanzen gewählt.

\Abs \Satz Vor der Wahl hat die Geschäftsführerin Finanzen dem StuRa zu erklären, dass ihr diese Finanzordnung bekannt ist.



\section{Aufgaben}

\Abs \Satz Die Geschäftsführerin Finanzen ist für die Einhaltung der einschlägigen Bestimmungen bei der Verwaltung der Einnahmen und Ausgaben der Studentenschaft verantwortlich.

\Abs \Satz Sie ist alleinig zur Erteilung von finanzwirksamen Anordnungen, insbesondere Kassenanweisungen, befugt, nicht aber alleinig zeichnungsberechtigt für die Konten der Studentenschaft.

\Abs \Satz Hält die Geschäftsführerin Finanzen durch Auswirkungen eines Beschlusses des StuRa die finanziellen oder wirtschaftlichen Interessen der Studentenschaft für gefährdet, so kann sie die Zahlung verweigern\. In diesem Falle muss der StuRa erneut über die Angelegenheit beraten\. Der daraus folgende Beschluss ist endgültig.

\Abs \Satz Die Geschäftsführerin Finanzen ist berechtigt die Kassen und Finanzbücher der Einrichtungen zu prüfen, an die Mittel der Studentenschaft weitergeleitet werden; dies trifft insbesondere auf die Fachschaften zu\. Sie kann bei Mängeln in der grundordnungs- bzw. ordnungsgemäßen Kassen- und Buchführung deren Berichtigung verlangen und, sollte diese nicht erfolgen, weitere Zahlungen zurückhalten\. Der StuRa ist darüber zu informieren.



\section{Bevollmächtigung von Vertreterinnen}

\Abs \Satz Die Geschäftsführerin Finanzen bevollmächtigt schriftlich in Abstimmung mit den anderen Geschäftsführerinnen des StuRa je zwei erste und zwei zweite Unterschriftsberechtigte für die Konten der Studentenschaft.

\Abs \Satz Die Bevollmächtigung endet
1. mit der von der Geschäftsführerin Finanzen gesetzten Frist,
2. durch schriftlichen Widerruf der Geschäftsführerin Finanzen,
3. durch Verzicht auf die Bevollmächtigung,
4. durch Tod, Verlust der Geschäftsfähigkeit und, bei Studentinnen, durch Exmatrikulation der Bevollmächtigten\. Hierüber sind unmittelbar und nachweisbar die kontoführenden Geldinstitute zu informieren.

\Abs \Satz Die Bevollmächtigten sind verantwortlich für alle Handlungen, die sie in Vertretung der Geschäftsführerin Finanzen ausüben.


\section*{3. Wirtschaftsplan}



\section{Grundlagen}

\Abs \Satz Der Wirtschaftsplan und dessen Nachträge werden unter Berücksichtigung des zur Erfüllung der Aufgaben notwendigen Bedarfs durch die Geschäftsführerin Finanzen für ein Wirtschaftsjahr aufgestellt und durch den StuRa beschlossen\. Er bildet die Grundlage der Verwaltung aller Erträge und Aufwendungen.

\Abs \Satz Der Wirtschaftsplan gliedert sich in einen Erfolgsplan und einen Finanzplan und zeigt die Entwicklung des Vermögens der Studentenschaft auf\. Der sich aus dem Erfolgsplan ergebende Überschuss/Fehlbetrag ist in den Finanzplan zu übernehmen.

\Abs \Satz Erträge und Aufwendungen sind getrennt voneinander in voller Höhe zu veranschlagen\. Es dürfen keine Erträge von Aufwendungen oder Aufwendungen von Erträgen vorweg abgezogen werden.

\Abs \Satz Für den gleichen Einzelzweck dürfen Gelder nicht an verschiedenen Stellen des Wirtschaftsplanes veranschlagt werden.

\Abs \Satz Der Wirtschaftsplan hat in Erträge und Aufwendungen ausgeglichen zu sein.



\section{Wirtschaftsjahr}

\Abs \Satz Das Wirtschaftsjahr beginnt in Abweichung von §~4~SäHO mit dem Sommersemester und endet mit Ablauf des darauffolgenden Wintersemesters.



\section{Veranschlagung der Erträge, Aufwendungen und Konten}

\Abs \Satz Der Wirtschaftsplan besteht aus Ertrags- und Aufwendungskonten mit jeweils fester Zweckbestimmung\. Die Erträge sind nach dem Entstehungsgrund, die Aufwendungen nach Zwecken getrennt zuzuordnen und, soweit erforderlich, zu erläutern\. Die Zuordnung ist so vorzunehmen, dass aus dem Wirtschaftsplan die Erfüllung der Aufgaben der Studentenschaft erkennbar ist\. In dem Wirtschaftsplan sind mindestens darzustellen: Erträge aus Studentenbeiträgen, wirtschaftlicher Tätigkeit, Entnahme aus Rücklagen, Überschuss des abgelaufenen Wirtschaftsjahres und Aufwendungen für Personal, Abschreibungen des Anlagevermögens, Büro- und Verbrauchsmaterial, Post und Kommunikation, Fahrtkosten, Rücklagenzuführung, wirtschaftliche Betätigung, Zuwendungen an Fachschaften und andere Stellen, Budgets der einzelnen Referate, Projekte, der Fehlbetrag des abgelaufenen Wirtschaftsjahres\. Stellen für Angestellte und deren Aufwendungen sind detailliert auszuweisen.

\Abs \Satz Die Konten sind mit einem Ansatzbetrag auszubringen\. Die Ansätze sind in ihrer voraussichtlichen Höhe zu bestimmen.

\Abs \Satz Der Titel Aufwandsentschädigungen muss mindestens nach Aufwandsentschädigungen für die Geschäftsbereiche aufgegliedert werden.

\Abs \Satz Die zur Zahlung von Aufwandsentschädigungen erforderliche Summe ist im Wirtschaftsplan zu veranschlagen und als solche zu kennzeichnen.



\section{Verwendung der Einnahmemittel}

\Abs \Satz Mittel, welche für andere Institutionen als die Studentenschaft ausgewiesen sind (Durchlaufposten), sind jeweils auf der Einnahmen- und Ausgabenseite in gleicher Höhe zu veranschlagen.

\Abs \Satz Die jeweiligen Fachschaftsmittel teilen sich in einen Sockel- und einen Pro-Kopf-Betrag gemäß der Beitragsordnung\. Solange das Guthaben einer Fachschaft sowohl mehr als 6.000 Euro als auch mehr als das Sechsfache ihrer Fachschaftsbeiträge beträgt, werden ihr keine Fachschaftsmittel für das aktuelle Semester überwiesen\. Diese Fachschaftsmittel verbleiben im Haushalt des Studentenrates.

\Abs \Satz Für Rücklagen zur Finanzierung von Großprojekten welche ein Finanzvolumen von 3.500~Euro überschreiten gilt Abs.~2~S.~2 insoweit nicht\. Die Mittel für derartige Projekte müssen auf separaten Konten ausgewiesen werden\. Die Höhe der Rücklagen muss in einem angemessenen Verhältnis zum Finanzvolumen des Großprojektes stehen\. Als angemessen ist dabei eine Deckung in Höhe von maximal 75~vom~Hundert der zu erwartenden Kosten anzusehen\. Großprojekte sind als solche dem GF Finanzen anzuzeigen.

\Abs \Satz Alle übrigen Einnahmen sind, soweit nicht anderweitig zweckbestimmt, grundsätzlich zur Deckung der Ausgaben des StuRa vorzusehen.



\section{Deckungsfähige Konten}

\Abs \Satz Ist eine genaue Veranschlagung in Konten ähnlicher Zweckbestimmung zum Zeitpunkt der Feststellung des Wirtschaftsplanes noch nicht übersehbar, so können diese Konten als ein- oder gegenseitig deckungsfähig ausgewiesen werden\. Dies hat im Wirtschaftsplan durch ausdrücklichen Vermerk zu geschehen.



\section{Nachtragswirtschaftsplan}

\Abs \Satz Die Änderung eines vom StuRa bereits rechtskräftig beschlossenen Wirtschaftsplanes ist nur durch einen Nachtragswirtschaftsplan möglich\. Bei dessen Aufstellung und Beschluss finden dieselben Bestimmungen Anwendung wie für die Aufstellung des Wirtschaftsplanes.



\section{Veröffentlichung}

\Abs \Satz Der beschlossene Wirtschaftsplan ist der Universitätsleitung zur Kenntnis zu bringen.

\Abs \Satz Der Wirtschaftsplan ist unverzüglich nach Beschlussfassung zu veröffentlichen.



\section{Inkrafttreten}

\Abs \Satz Der Wirtschaftsplan tritt am Tage nach seiner Bekanntmachung, frühestens jedoch mit Beginn des Wirtschaftsjahres, für das der Wirtschaftsplan aufgestellt worden ist, in Kraft.



\section{Bedeutung des Wirtschaftsplanes gegenüber Dritten}

\Abs \Satz Durch den Wirtschaftsplan werden Ansprüche oder Verbindlichkeiten gegenüber Dritten weder begründet noch aufgehoben.



\section{Vorläufige Wirtschaftsführung}

\Abs \Satz Grundlage für die Wirtschaftsführung vor Inkrafttreten des Wirtschaftsplanes sind die Ansätze des Vorjahres, von diesen darf für jeden Monat ein Zwölftel in Anspruch genommen werden.

\Abs \Satz Sieht der Entwurf des Wirtschaftsplanes niedrigere Ansätze gegenüber dem Vorjahr vor, so ist bei der vorläufigen Wirtschaftsführung von diesen auszugehen.

\Abs \Satz Neue Konten dürfen erst nach Inkrafttreten des Wirtschaftsplanes in Anspruch genommen werden.



\section{Rücklagen}

\Abs \Satz Im Wirtschaftsplan werden Rücklagen vorgesehen.

\Abs \Satz Die Höhe sollte 20~\% des Gesamtkapitals des StuRa nicht überschreiten.

\Abs \Satz Die Entwicklung der Rücklagen ist im Geschäftsbericht als Anlage zur Bilanz zu begründen.


\section*{4. Kassenwesen}



\section{Verwaltung der Mittel durch Fachschaften}

\Abs \Satz Bei der Bewirtschaftung von Studentenschaftsmitteln durch die Fachschaften ist ein Nachweis zu führen, aus dem sich die Erträge und die Aufwendungen ergeben\. Die Buchungen sind zu belegen, die einschlägigen Bestimmungen der Finanzordnung gelten entsprechend\. Eine Kassenprüfung ist in der Satzung oder Finanzordnung einer Fachschaft vorzusehen\. Am Ende des Wirtschaftsjahres nicht verbrauchte Mittel sind im folgenden Wirtschaftsjahr als Erträge zu verbuchen.

\Abs \Satz Existiert in einer Fachschaft kein Fachschaftsrat, so werden die Mittel der betreffenden Fachschaft behelfsweise vom StuRa für die Dauer von zwei Semestern verwaltet\. Diese Fachschaftsgelder sind unverzüglich weiterzuleiten, sobald die Gründung eines Fachschaftsrates erfolgt\. Sollte sich nach Ablauf der Verwaltungsfrist kein FSR gebildet haben, so fließen diese Mittel der Studentenschaft zu.

\Abs \Satz Jede Fachschaft muss jährlich zum 31.~März für das vergangene Jahr dem StuRa einen Sachanlagennachweis erbringen\. Die Kassenbestände müssen ebenfalls jährlich zum 31.~März dem StuRa gemeldet werden.

\Abs \Satz Innerhalb des Jahres prüft die Geschäftsführerin Finanzen mindestens einmal die Finanzen der einzelnen Fachschaften auf ihre Ordnungsmäßigkeit\. Das Ergebnis der Prüfung ist zu vermerken\. Die Auszahlung neuer Fachschaftsgelder erfolgt erst nach erfolgreicher Prüfung.



\section{Außerplanmäßige Ausgaben}

\Abs \Satz Außerplanmäßige Aufwendungen, die über den Ansatz eines Kontos hinausgehen oder unter keine Zweckbestimmung des Wirtschaftsplanes fallen, dürfen erst geleistet werden, wenn ein entsprechender Nachtrag zum Wirtschaftsplan in Kraft getreten ist\. Dies gilt nicht für unabweisbare Aufwendungen, insbesondere für Aufwendungen, die zur sparsamen Fortführung der Verwaltung erforderlich sind, sofern Mehraufwendungen an anderer Stelle des Wirtschaftsplanes eingespart werden\. Die Geschäftsführerin hat dem StuRa hiervon unverzüglich Kenntnis zu geben\. Bei der Aufstellung eines Nachtragswirtschaftsplanes haben diese Aufwendungen Vorrang.



\section{Einhaltung des Wirtschaftsplanes}

\Abs \Satz Aufwendungen sind nur in Übereinstimmung mit der Zweckbindung der Konten auszugeben\. Ist die Zuordnung von Aufwendungen zweifelhaft, so hat die Verbuchung in einem der sich anbietenden Konten zu erfolgen\. Eine Verbuchung unter verschiedenen Konten ist unzulässig.



\section{Vorausleistungen}

\Abs \Satz Leistungen der Studentenschaft vor Empfang der Gegenleistung dürfen nur vereinbart werden, sofern dies im allgemeinen Geschäftsverkehr üblich oder durch besondere Umstände gerechtfertigt ist.



\section{Verantwortlichkeit}

\Abs \Satz Für das ordnungsgemäße Kassenwesen des StuRa ist die Geschäftsführerin Finanzen verantwortlich.



\section{Kassenverwalterin}

\Abs \Satz Der StuRa ernennt eine Kassenverwalterin und eine Vertreterin\. Die zu ernennende Person muss die zur Erfüllung dieser Aufgaben notwendigen Fachkenntnisse der Buchhaltung und Finanzwirtschaft nachweisen und darf nicht Mitglied des StuRa sein.

\Abs \Satz Die Kassenverwalterin hat folgende Aufgaben:
1. Vornahme von Buchungen und Sammlung der Belege,
2. Verwaltung der Konten,
3. Entgegennahme und Auszahlung von Bargeld,
4. Verwahrung der Bargeldbestände, Wertvordrucke und -gegenstände,
5. Erstellung von Jahresabschluss und Übersichten sowie
6. Vorlage einer nach dem Wirtschaftsplan gegliederten Übersicht über die Erträge und Aufwendungen eines jeden Monats für die Geschäftsführerin Finanzen.

\Abs \Satz Die Kassenverwalterin ist nicht berechtigt, ohne ausdrückliche Erlaubnis der Geschäftsführerin Finanzen Auskünfte an Dritte über die Finanzgeschäfte und -bestände des StuRa zu erteilen.



\section{Zahlungsverkehr}

\Abs \Satz Der Zahlungsverkehr wird bar und über die Konten des StuRa abgewickelt.

\Abs \Satz Der Bargeldbestand soll höchstens 1000~Euro betragen\. Bei Überschreitung dieser Summe ist spätestens am nächsten Werktag auf die Konten des StuRa einzuzahlen. Barbestände sind in Geldkassetten und im Stahlschrank sicher aufzubewahren.

\Abs \Satz Überweisungsaufträge, Scheckhefte, Kontenkarten sind gleichfalls von der Kassenverwalterin sicher unter Verschluss zu halten.

\Abs \Satz Die Kassenverwalterin hat den Kontenstand mindestens einmal monatlich zu ermitteln und dem Sollbestand gegenüberzustellen\. Es ist sichtbar zu machen, wie sich der Finanz-Istbestand aus Bargeld und Kontenguthaben zusammensetzt.

\Abs \Satz Belege, Kassenbücher und Kontoauszüge sind nach Abschluss des Wirtschaftsjahres nach den gesetzlichen Bestimmungen aufzubewahren.



\section{Kassenführung}

\Abs \Satz Auszahlungen dürfen nur von der Kassenverwalterin und nur auf Grund schriftlicher Anordnungen veranlasst werden.

\Abs \Satz Über jede Bareinzahlung ist der Einzahlerin eine Quittung zu erteilen, soweit der Zahlungsnachweis nicht in anderer Form sichergestellt ist\. Über jede Barauszahlung ist von dem Empfänger eine Quittung zu verlangen.



\section{Kassenanordnungen}

\Abs \Satz Kassenanordnungen sind von der Geschäftsführerin Finanzen zu unterzeichnen\. Mit der Unterzeichnung übernimmt die Geschäftsführerin Finanzen die Verantwortung dafür, dass
1. keine offensichtlich erkennbaren Fehler in der Kassenanordnung enthalten sind,
2. die sachliche und rechnerische Richtigkeit der in der Kassenanordnung enthaltenen Angaben bescheinigt worden ist,
3. das Konto richtig bezeichnet wurde,
4. Ausgabemittel in der vorgegebenen Höhe zur Verfügung stehen\.
Die Kassenanordnung muss gegebenenfalls im Zusammenhang mit den beigefügten Unterlagen Zweck und Anlass der Zahlung begründen und eine Prüfung ohne Rückfragen ermöglichen.

\Abs \Satz Die eine Einnahme oder Ausgabe begründenden Teile einer Kassenanordnung bedürfen der Feststellung der sachlichen und rechnerischen Richtigkeit\. Die Feststellung der sachlichen Richtigkeit obliegt einer Geschäftsführerin, die der rechnerischen Richtigkeit einer zu bevollmächtigenden Angestellten.



\section{Buchführung}

\Abs \Satz Über die Zahlungen ist sowohl nach der Zeitfolge als auch nach der im Wirtschaftsplan vorgesehenen Kontenordnung Buch zu führen\. Die Buchführung erfolgt nach kaufmännischen Grundsätzen und in Anlehnung an den DATEV-Kontenrahmenplan~SKR\. Zahlungen sind für das Wirtschaftsjahr zu buchen, in dem sie eingegangen oder geleistet worden sind.

\Abs \Satz Es ist eine doppelte Buchführung zu sichern, die aus Grund- und Hauptbuch besteht\. Der Kontenplan ergibt sich aus dem abgeleiteten Kontenrahmenplan i.~V.~m. dem Wirtschaftsplan; die Konten sind zum Ende des Wirtschaftsjahres zur Jahresabschlussrechnung abzuschließen.



\section{Abschreibung}

\Abs \Satz Für aus Studentenschaftsmitteln angeschaffte Vermögensgegenstände, die nicht zum Verbrauchsmaterial gehören, ist ein Konto "`Abschreibung"' zu führen\. Entsprechend der zu erwartenden Nutzungsdauer und den amtlichen AfA-Tabellen folgend sind die Vermögensgegenstände linear abzuschreiben, bis der Buchwert Null erreicht ist.

\Abs \Satz Die Sammlung der Abschreibungsbeträge ist als liquider Bestand in der Vermögensübersicht (Anlage zum Wirtschaftsplan) darzustellen.



\section{Inventarverzeichnis}

\Abs \Satz Die Kassenverwalterin hat ein Inventarverzeichnis zu führen\. Darin sind alle Vermögensgegenstände aufzuführen, deren Anschaffungswert 150~Euro ohne Mehrwertsteuer übersteigt und die nicht zum Verbrauchsmaterial gehören.

\Abs \Satz Rechnungen aller inventarisierten Gegenstände sind in der Reihenfolge der Anschaffung zu nummerieren und zu archivieren.

\Abs \Satz Die Entfernung eines beim StuRa inventarisierten Gegenstandes ist aktenkundig zu begründen.

\Abs \Satz Im Rahmen der jährlichen Abschlussprüfung und vor Übergabe der Geschäfte der Geschäftsführerin Finanzen an eine Nachfolgerin ist das Inventar zu überprüfen, eine Liste eventuell abhanden gekommener Vermögensteile ist zu erstellen und von der Geschäftsführerin Finanzen zu unterschreiben\. Der StuRa ist darüber zu informieren.

\Abs \Satz Die Inventur ist aktenkundig festzuhalten.



\section{Rechnungslegung}

\Abs \Satz Unverzüglich zum Ende des Wirtschaftsjahres stellt die Kassenverwalterin den Jahresabschluss auf der Grundlage der Buchführung in Form der Bilanz mit Gewinn und Verlustrechnung sowie Geschäftsbericht auf.

\Abs \Satz Alle Erträge und Aufwendungen des abgeschlossenen Wirtschaftsjahres sind im Rechnungsergebnis auszuweisen\. Der sich ergebende Überschuss bzw. Fehlbetrag ist zu kennzeichnen.

\Abs \Satz Vereinnahmte Beträge, die zurückgezahlt werden müssen, sind als Verbindlichkeiten auszuweisen; verausgabte Beträge, die zur Rückzahlung offen stehen, sind als Forderungen zu erfassen.

\Abs \Satz Dem Rechnungsergebnis sind beizufügen:
1. ein Nachweis über im Wirtschaftsplan nicht vorgesehene Einnahmen, insbesondere solche aus der Veräußerung von Sachen oder Rechten der Studentenschaft sowie
2. eine Vermögensübersicht der Gliederung nach §~266~HGB.

\Abs \Satz Das Rechnungsergebnis ist den Prüfern gemäß §~42~Abs.~1 unverzüglich zuzuleiten\. Sollten aus dem Rechnungsergebnis Verstöße gegen die Finanzordnung oder übergeordnete Bestimmungen sichtbar werden, so ist der StuRa hierüber zu informieren.



\section{Sicherung der wirtschaftlichen Verwahrung}

\Abs \Satz Der für Aufwendungen nicht erforderliche Finanzbestand ist so anzulegen, dass ein Verlust ausgeschlossen ist und im Bedarfsfall jederzeit über die Guthaben der Studentenschaft verfügt werden kann.

\Abs \Satz Zur Vermeidung einer Verminderung der Guthaben durch die Geldentwertung sind längerfristig nicht benötigte Geldmittel entsprechend anzulegen. Eine Anlage in risikobehaftete Wertpapiere o.ä. ist unzulässig.



\section{Begleichung von Rechnungen}

\Abs \Satz Vor der Begleichung sind Rechnungen durch die Geschäftsführerin Finanzen auf ihre Richtigkeit zu prüfen\. Lieferscheine sind mit der zugehörigen Rechnung aufzubewahren.

\Abs \Satz Rechnungen sind nicht vor Zahlungsziel zu begleichen\. Skontofristen sind dabei jedoch zu beachten.



\section{Anschaffung und Veräußerung von Eigentum}

\Abs \Satz Vor der Anschaffung von Gegenständen sind die allgemeinen Vergaberichtlinien zu beachten\. Die Auswahl hat mit Begründung aktenkundig zu erfolgen.

\Abs \Satz Gegenstände, die sich im Eigentum der Studentenschaft befinden und noch einen Restwert besitzen, dürfen nur auf Beschluss des StuRa und zum tatsächlichen Wert veräußert werden\. Hierbei sind Angebote von Kaufinteressentinnen einzuholen. Gegenstände, die abgeschrieben sind, dürfen zum tatsächlichen Wert von der Geschäftsführung veräußert werden\. Aussonderungen sind dem StuRa vier Vorlesungswochen vorher anzuzeigen\. Veräußerte Gegenstände müssen aus dem Inventarverzeichnis entfernt und aktenkundig begründet werden\. Die einzelnen Fachschaften entscheiden selbst in kompetenter und angemessener Form über die Veräußerung ihrer Sachmittel.

\Abs \Satz Von diesen Bestimmungen kann bei laufenden Geschäften oder geringem finanziellen Umfang abgewichen werden.


\section*{5. Bewilligung von Zahlungen}



\section{Anmeldepflicht von Ausgaben}

\Abs \Satz Ausgaben sowie Aufträge im Namen und auf Rechnung der Studentenschaft bedürfen der Anmeldung bei der Geschäftsführerin Finanzen, soweit sie nicht durch sie selbst angeordnet wurden.

\Abs \Satz Sieht die Geschäftsführerin Finanzen angezeigte Ausgaben als nicht notwendig oder mit den Aufgaben der Studentenschaft nicht vereinbar an, so kann sie im Einvernehmen mit den anderen Geschäftsführerinnen des StuRa die Unterlassung verlangen\. Eine solche Entscheidung ist zu begründen.

\Abs \Satz Werden Ausgaben nicht binnen vier Monaten nach ihrer Anzeige getätigt, gelten sie als nicht angezeigt\. Diese Frist kann durch die Geschäftsführerin Finanzen verlängert werden.



\section{Studentische Projekte}

\Abs \Satz Ist dies im Wirtschaftsplan vorgesehen, können studentische Projekte finanziell unterstützt werden, sofern sie den satzungsmäßigen Aufgaben der Studentenschaft entsprechen.

\Abs \Satz Über die Förderung entscheidet der StuRa auf Antrag\. Sie erfolgt zweckgebunden.



\section{Bürgschaften und Darlehen}

\Abs \Satz Bürgschaften und Garantien in Verträgen dürfen nicht übernommen, Darlehen nicht gewährt werden\. Ausnahmen regelt die Finanzierungsrichtlinie

\Abs \Satz Der StuRa kann abweichend hiervon zur Abwendung einer Mitgliedern der Studentenschaft drohenden Notlage die Übernahme einer Bürgschaft mit Mehrheit der Mitglieder beschließen.



\section{Längerfristige Verpflichtungen}

\Abs \Satz Maßnahmen, die die Studentenschaft zu Ausgaben in künftigen Wirtschaftsjahren verpflichten können, sind nur zulässig, wenn der StuRa dies mit \nicefrac{2}{3}~Mehrheit beschließt\. Dies gilt nicht für die laufenden Geschäfte oder für Verpflichtungen deren finanzielle Auswirkungen gering sind.



\section{Beitragspflichtige Mitgliedschaft}

\Abs \Satz Eine Mitgliedschaft der Studentenschaft in einem Verein oder einer anderen Institution, die zur Zahlung von Beiträgen verpflichtet, ist nur zulässig, wenn der StuRa mit der Mehrheit seiner Mitglieder zustimmt oder der Beitrag 150~Euro jährlich nicht übersteigt.

\Abs \Satz Unzulässig ist eine Mitgliedschaft in Vereinen oder Institutionen, deren Ziele den satzungsmäßigen Aufgaben der Studentenschaft entgegenstehen.



\section{Ausgaben von erheblicher Höhe}

\Abs \Satz Angelegenheiten von erheblicher finanzieller oder grundsätzlicher Bedeutung sowie über- und außertarifliche Leistungen und Verfügungen über das Vermögen bedürfen, soweit nicht bereits im Wirtschaftsplan so vorgesehen, der Zustimmung des StuRa mit Mehrheit der Mitglieder.



\section{Reisekosten}

\Abs \Satz Reisekosten können erstattet werden, wenn ein nachweisbarer Nutzen für die studentische Selbstverwaltung oder die Studentenschaft aus der Reise erwächst.

\Abs \Satz Die Reise beginnt und endet an der Wohnung des Studienortes\. Muss die Reise an einer anderen Stelle angetreten oder beendet werden, kann diese an die Stelle der Wohnung treten.

\Abs \Satz Reisekosten sind binnen zwei Wochen nach Beendigung der Reise bei der Kassenwärtin abzurechnen\. Grundlage für die Rückerstattung von Auslagen (z.~B. Fahrkarten, Übernachtungsrechnungen, Teilnehmergebühren) sind die Originalbelege.

\Abs \Satz Bevorzugt sollen öffentliche Verkehrsmittel benutzt werden.

\Abs \Satz Soweit Beförderungsmittel nicht mit dem Semesterticket oder sonstigen Freifahrtberechtigungen benutzt werden können, werden für Fahrten, die mit regelmäßig verkehrenden Beförderungsmitteln erfolgen, grundsätzlich maximal die Kosten der günstigsten benutzbaren Fahrkarte erstattet\. Fahrscheine sind nach Möglichkeit so zeitig zu beschaffen, dass Frühbucherinnenrabatte in Anspruch genommen werden können. Mitfahrerinnenrabatte sind zu nutzen.

\Abs \Satz Bei Fahrten mit der Bahn, deren Ziel außerhalb des Freistaates Sachsen liegt, können bei Nutzung von Zügen der DB AG auch Züge des Fernverkehrs genutzt werden\. Bei Fahrten die innerhalb des Freistaates Sachsen, des Freistaates Thüringen und des Landes Sachsen-Anhalt enden, ist, soweit möglich Sonderangebote des Nahverkehrs zu berücksichtigen\. Bei Nachtfahrten werden die Kosten für den Liegewagen erstattet, wenn die Fahrt vor 23.00~Uhr angetreten und nach 4.00~Uhr beendet wurde.

\Abs \Satz Der StuRa kann die Kosten eines gültigen Ermäßigungsausweises (z.~B.~Bahn-Card) rückwirkend übernehmen, wenn durch dessen Gebrauch die Ersparnis an Reisekosten den Anschaffungspreis übersteigt\. Dabei werden alle entsprechenden Fahrten berücksichtigt, die seit der ersten Fahrt für den StuRa bzw. seit dem mit Ablauf des letzten durch den StuRa bezahlten Ermäßigungsausweises angefallen sind\. Der Antrag auf Erstattung eines Ermäßigungsausweises muss bis spätestens einen Monat nach Ablauf desselben gestellt worden sein.

\Abs \Satz Bei  Benutzung  privater   Kraftfahrzeuge  erfolgt  eine  Erstattung  der Reisekosten in Höhe von 0,17 Euro pro Kilometer (+ 0,02 Euro pro Kilometer für jede mitgenommene Person), jedoch nicht  mehr  als  der  günstigste Fahrschein (bei DB AG Normalpreis Produktklasse~C auf kürzester Wegstrecke mit BahnCard 50)in der 2.~Wagenklasse der DB AG bzw. eines anderen EVU\. Bei der Benutzung privater Kraftfahrzeuge aus triftigen Gründen erfolgt eine Erstattung der Reisekosten in Höhe von 0,30 Euro pro Kilometer (+ 0,02 Euro pro Kilometer für jede mitgenommene Person)\. Ob derartige Gründe vorliegen entscheidet die Geschäftsführung, das Plenum bzw. der Förderausschuss zum Zeitpunkt der Antragstellung\. Im Nachhinein können triftige Gründe nicht mehr geltend gemacht werden. \
 
(8a) \Satz Stimmberechtigte Plenumsmitglieder des StuRa der TUD können für die Teilnahme an ordentlichen, wie außerordentlichen Plenumssitzungen des StuRa mit einem privaten KFZ anreisen, wenn sowohl deren entsendende Fachschaft, als auch deren Wohnsitz über 50 km von der Besucheranschrift des StuRa der TUD entfernt sind\. Für die Nutzung eines KFZs liegt ein triftiger Grund vor\. Es werden 0,30 Euro pro km erstattet. Grundlage für die Berechnung der Streckenlänge ist die Strecke mit der kürzesten Fahrzeit (unabhängig von der Verkehrs-situation), gemäß der Routenberechnung von Google Maps oder einem ähnlichen Dienst\. Die Abrechnung hat bei der Kassenwärtin des StuRa der TUD binnen von drei Monaten zu erfolgen\. Der Ausdruck der Routenberechnung ist bei der Abrechnung vorzulegen\. Abs. 3 Satz 1, Abs. 5 und Abs. 8 des § 38 der Finanzordnung des StuRa finden hier keine Anwendung.

\Abs \Satz Für Übernachtungen werden maximal die Kosten der preiswertesten und in zumutbarer Entfernung liegenden Jugendherberge getragen\. Die Übernachtungskosten werden nicht gezahlt, wenn die Reise vor 2.00~Uhr des darauffolgenden Tages endet bzw. enden könnte oder insgesamt weniger als acht Stunden dauert. 

\Abs \Satz Tagegeld in der Höhe von 6~Euro kann gewährt werden, wenn die Dienstreise länger als 16~Stunden pro Tag dauert, zwischen mindestens acht und 16~Stunden Abwesenheit in halber Höhe\. Wird kostenlos oder über den Tagungsbeitrag finanzierte Verpflegung bereitgestellt, so vermindert sich das Tagegeld für das Frühstück um 20~\%, für das Mittag um 30~\% und für das Abendbrot um 50~\% des Tagegeldes.

\Abs \Satz Tagungskosten können bis zur vollen Höhe übernommen werden.


\section{Bewirtungen}

\Abs \Satz Bewirtungen auf Rechnung der Studentenschaft sind nur zulässig, wenn sie sich aus den Aufgaben der Studentenschaft ergeben.

\Abs \Satz Eine Erstattung von Bewirtungskosten ist nur zulässig, wenn Belege über die Ausgaben und eine Liste der bewirteten Personen vorliegen.


\section*{6. Aufwandsentschädigungen}



\section{Allgemeines zu Aufwandsentschädigungen}

\Abs \Satz Aufwandsentschädigungen (AE) sollen für die Zeit entschädigen, in der andere Studentinnen arbeiten gehen können\. Sie sind keine Gehaltszahlungen.

\Abs \Satz Berechtigt zum Erhalt von AE sind die Geschäftsführerinnen, Referenteninnen, Referatsmitarbeiterinnen, Mitglieder des Sitzungsvorstands, Mitarbeiter an Projekten des Studentenrates und die studentischen Sportobleute des Universitätssportzentrums der TU Dresden.



\section{Zahlung der Aufwandsentschädigungen}

\Abs \Satz AE müssen binnen 10~Tagen nach dem Anspruchszeitraums bei der Geschäftsführung beantragt werden\. Die Höhe der AE wird von der Geschäftsführung beraten und beschlossen.

\Abs \Satz Für die Höhe der AE gilt folgender Maßstab\. 40~Monatswochenstunden werden mit 600~Euro entschädigt.

\Abs \Satz Die studentischen Sportobleute des Universitätssportzentrums der TU Dresden können eine AE in Höhe von maximal 200~Euro pro Person und Semester erhalten.

\Abs \Satz Als Anspruchszeitraum gilt jeweils genau ein Kalendermonat\. Für die Sportobleute gilt als Anspruchszeitraum ein Semester.

\Abs \Satz Die Höhe der Aufwandsentschädigung, die vom StuRa gezahlt wird, ist auf 300~Euro pro Person und Monat begrenzt.

\Abs \Satz AE nach Abs.~1 werden binnen fünf Werktagen nach Ende des Anspruchszeitraums, AE nach Abs.~2 binnen fünf Tagen nach Bewilligung ausgezahlt.


\section*{7. Prüfungswesen}



\section{Verfahren der Prüfung}

\Abs \Satz Der Jahresabschluss der Studentenschaft wird durch die Innenrevision der Universität geprüft.

\Abs \Satz Die Kassenprüfung ist mindestens einmal jährlich unangemeldet durchzuführen mit dem Zweck
1. Kassenist- und -sollbestand festzustellen und
2. die Buchhaltung sowie die Belegerfassung zu kontrollieren.

\Abs \Satz Darüber hinaus steht es den Prüferinnen frei sich zu vergewissern, ob die gesetzlichen Bestimmungen der Wirtschaftsführung sowie dieser Finanzordnung eingehalten wurden.

\Abs \Satz Über die Kassenprüfung ist von den Prüferinnen ein Testat anzufertigen.

\Abs \Satz Stellt die Prüferin Mängel fest, so kann sie deren Beseitigung von der Kassenverwalterin und der Geschäftsführerin Finanzen innerhalb von 14~Tagen verlangen. Danach ist eine erneute Prüfung durchzuführen\. Bei erheblichen Mängeln, die zur Verweigerung des Testats führen, ist der StuRa unverzüglich in Kenntnis zu setzen und verpflichtet geeignete Maßnahmen zu treffen.

\Abs \Satz Kassenverwalterin und Geschäftsführerin Finanzen sind verpflichtet, zur Prüfung anwesend zu sein\. Sie haben Fragen der Prüferin gewissenhaft und ehrlich zu beantworten.



\section{Bekanntgabe des Prüfungsergebnisses}

\Abs \Satz Das Testat der Prüfung ist dem StuRa zur Kenntnis zu geben\. Legt die Geschäftsführerin Finanzen Widerspruch gegen den Prüfbericht ein, kann der StuRa beschließen, dass eine weitere Prüfung durchgeführt wird, zu der auch ein anderer Prüfer bestellt werden kann\. Das Prüfungsergebnis kann der Hochschulleitung zur Kenntnis übergeben werden.



\section{Jahresabschlussprüfung}

\Abs \Satz Unverzüglich nach der Feststellung des Rechnungsergebnisses ist eine Jahresabschlussprüfung durchzuführen\. Zeitgleich erfolgt eine Inventur der Vermögensbestände\. Die richtige Übertragung des Überschusses oder Fehlbetrages ist zu bescheinigen.

\Abs \Satz §~42~und~§~43 gelten entsprechend.

\section{Rechnungsprüfung durch staatliche Stellen}

\Abs \Satz Die Wirtschaftsführung der Studentenschaft unterliegt der Prüfung durch den Landesrechnungshof und der Vorprüfung durch die Innenrevision der Universität.

\end{multicols}

\nopagebreak
\vspace{1cm}
Inkraftgetreten am 22.~Juni~2001.
\\


\footnotesize
Geändert am 07.~Februar~2003\\
§~33 Reisekostenregelung wegen Änderungen Preissystem der DB~AG.

Geändert am 13.~Dezember~2007\\
§~5: In Abs. 1 "`durch den Geschäftsführer Finanzen erstellt und"' eingefügt und "`Einnahmen und Ausgaben"' ersetzt durch "`Erträge und Aufwendungen"'. Abs. 2 eingefügt. In Abs. 3 und 5 "`Einnahmen und Ausgaben"' ersetzt durch "`Erträge und Aufwendungen"'.\\
§~7: Umbenannt in "`Veranschlagung der Erträge, Aufwendungen und Konten"'. In Abs. 1 "`Einnahme- und Ausgabekonten"' durch "`Ertrags- und Aufwendungskonten"', "`Einnahmen"' durch "`Erträge"' und "`Ausgaben"' ersetzt durch "`Aufwendungen"'. In Nr. 2 "`Büro- und  Geschäftsausstattung"' gestrichen und "`Abschreibungen des Anlagevermögens"' eingefügt. In Satz 6 "`und deren Aufwendungen"' eingefügt. Abs. 2 Satz 3 gestrichen.\\
§~8: In Abs. 2 Satz 2 ersetzt durch "`Solange das Guthaben einer Fachschaft sowohl mehr als 6.000 Euro als auch mehr als das Sechsfache ihrer Fachschaftsbeiträge beträgt, werden ihr keine Fachschaftsmittel für das aktuelle Semester überwiesen."'. NEU: Satz 3.\\
§~11: Abs. 2 ersetzt durch "`Der Wirtschaftsplan ist unverzüglich nach Beschlussfassung zu veröffentlichen."'\\
§~14: In Abs. 1 "`Der zu Auszahlungen"' ersetzt durch "`Der für Aufwendungen"'.\\
§~15: Umbenannt in "`Außerplanmäßige Aufwendungen"'. In Satz 1 "`Ausgaben"' durch "`Außerplanmäßige Aufwendungen"' ersetzt. In Satz 2 "`Mehrausgaben"' durch "`Mehraufwendungen"' ersetzt. In Satz 4 "`Ausgaben"' durch "`Aufwendungen"' ersetzt.\\
§~16: In Abs. 1 "`Einnahmen und die Ausgaben"' durch "`Erträge und die Aufwendungen"' sowie "`Einnahme"' durch "`Erträge"' ersetzt. NEU: Abs. 3 und 4.\\
§~17: "`Ausgabemittel"' und "`Ausgaben"' jeweils ersetzt durch "`Aufwendungen"'.\\
§~22: "`Einnahmen und Ausgaben"' ersetzt durch "`Erträge und Aufwendungen"'.\\
§~29: "`Einnahmen und Ausgaben"' ersetzt durch "`Erträge und Aufwendungen"'.\\
§~33: In Abs. 6 Satz 3 ersetzt durch "`Bei Nachtfahrten werden die Kosten für den Liegewagen erstattet, wenn die Fahrt vor 23.00 Uhr angetreten und nach 4.00 Uhr beendet wurde."'. Gestrichen: Abs. 12.\\
§~39: "`BAT-O"' ersetzt durch "`dem TVL Tarifgebiet Ost"'.\\
§~40: Abs. 2 ersetzt durch "`Gegenstände, die sich im Eigentum der Studentenschaft befinden und noch einen Restwert  besitzen, dürfen nur auf Beschluss des StuRa und zum tatsächlichen Wert veräußert werden. Hierbei sind Angebote von Kaufinteressenten einzuholen. Gegenstände, die abgeschrieben sind, dürfen zum tatsächlichen Wert von der Geschäftsführung veräußert werden. Aussonderungen sind dem Stura vier Vorlesungswochen vorher anzuzeigen. Veräußerte Gegenstände müssen aus dem Inventarverzeichnis entfernt und aktenkundig begründet werden.Die einzelnen Fachschaften entscheiden selbst in kompetenter und angemessener Form über die Veräußerung ihrer Sachmittel."'.

Geändert am 17.~Juli~2008\\
Umsortierung der Reihenfolge der Paragraphen;\\
§ 7 Abs. 3 und 4 NEU;\\
alt § 31 gestrichen;\\
§ 34, alt § 36 in Abs. 1 "`, Darlehen nicht gewährt werden. Ausnahmen regelt die Finanzierungsrichtlinie."' eingefügt, Abs. 3 gestrichen;\\
§ 38, alt § 33 Abs. 3 "`Finanzreferentin"' durch "`Kassenwärtin"' ersetzt;\\
alt § 39 gestrichen;\\
§§ 40 und 41 NEU;\\
alt Abschnitt 8, § 48 gestrichen;

Geändert am 12.~August~2010\\
Anpassung aufgrund des Wegfalls der Bindung der Studentenschaft an die Reisekosten-Ordnung der TU Dresden;\\
§ 38 Abs. 8 NEU;\\


Geändert am 25. Oktober 2013\\
§ 22 Abs. 2 Änderung des "`Bargeldbestand"' auf "`1000 Euro"';\\
§ 38 Abs. 8 "`Reisekosten in Höhe von 0,22 Euro pro Kilometer"' ersetzt durch "`Reisekosten in Höhe von 0,17 Euro pro Kilometer (+ 0,02 Euro pro Kilometer für jede mitgenommene Person)"'; "`BahnCard"' ersetzt durch "`BahnCard 50"';\\
§ 38 Abs. 8 Satz 2,3,4 NEU;\\
§ 38 Abs. 8a NEU;\\

\normalsize
~\\*[4cm]
\begin{center}
\hspace*{\fill}
\parbox{7cm}{Jessica Rupf\\GF Soziales}
\hfill\parbox{7cm}{Matthias Funke\\GF Finanzen}
\hspace*{\fill}
\end{center}