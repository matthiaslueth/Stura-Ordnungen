%\addchap[Social Media-Richtlinie des Studentenrates der TU~Dresden]{Social Media-Richtlinie\\des Studentenrates der TU Dresden}
\markright{Social Media Ordnung}
\setcounter{section}{0}
\begin{multicols}{2}

\section{Präambel}
\Abs \Satz Sämtliche Normierungen, bei denen der StuRa Adressat ist, sind nur für diesen einschlägig. \Satz Fachschaftsräte können davon abweichen.

\section{Begriffsbestimmungen}
\Abs \Satz Soziale Medien sind digitale Plattformen, die der gegenseitigen Kommunikation und dem interaktiven Austausch von Informationen dienen.

\Abs \Satz Diese sind abzugrenzen von
\begin{enumerate}
\item traditionellen Massenmedien, die vorrangig auf die Verbreitung von Informationen abzielen. 
\item internen Arbeitsmedien, die exklusiv für Mitarbeiterinnen des StuRa zur Verfügung stehen.
\end{enumerate}

\section {Soziale Medien}
\Abs \Satz Der StuRa betreibt und verwaltet soziale Medien als soziale Medien des StuRa oder partizipiert an sozialen Medien im Auftrag des Plenums oder der Geschäftsführung.

\Abs \Satz Die Administration obliegt der Geschäftsführung und die Referentin Öffentlichkeitsarbeit\. Mitarbeiterinnen des StuRas haben die Möglichkeit mit Zustimmung der Geschäftsführung als Redakteurinnen tätig zu sein\. Das Plenum ist über personelle Änderungen in Kenntnis zu setzen.

\Abs \Satz Soziale Medien dienen der Unterstützung der Weitergabe von Informationen des StuRas.

\Abs \Satz Die sozialen Medien müssen Rahmenbedingungen bereitstellen, die die Erfüllung von §3 (1) ermöglichen.

\Abs \Satz Die Autorenschaft veröffentlichter Beiträge ist für die gesamte Nutzerschaft klar zu kennzeichnen.

\section {Inhalte sozialer Medien}
\Abs \Satz Die mittels sozialen Medien verbreiteten Inhalte sollen im Allgemeinen öffentlich zugänglich sein\. Die interaktive Teilnahme von anderen Benutzern der sozialen Medien soll ermöglicht werden.

\Abs \Satz Die mittels sozialen Medien verbreiteten Inhalte dienen den folgenden Aufgaben:
\begin{enumerate}
\item Repräsentation des StuRas
\item Weitergabe von Informationen im Rahmen der Tätigkeiten des StuRas und dessen Strukturen
\item Erfüllung der Aufgaben der verfassten Studentenschaft nach §2 (1) der Grundordnung der Studentenschaft der TU Dresden
\end {enumerate}
\Abs \Satz Nicht beworben werden dürfen Veranstaltungen, Artikel oder politische Ideen, solange der StuRa diese nicht unterstützt\. Grundsätzlich können Veranstaltungen von der TU Dresden und dem Studentenwerk Dresden beworben werden.

\Abs \Satz Interaktionen rassistischer, nationalistischer, antisemitischer und menschenverachtender Natur sollen unterbunden werden. 

\Abs \Satz Das Veröffentlichen, Verändern und Löschen von Inhalten ist zu dokumentieren.

\Abs \Satz Sachverhalte, die personenbezogene und schützenswerte Daten enthalten, dürfen nicht über soziale Medien ausgetauscht werden.

\end{multicols}

\textbf{Änderungsanträge}\\
Änderungsantrag 1a (Antragssteller)\\
§2 (5) Die Autorenschaft veröffentlichter Beiträge ist öffentlich zu kennzeichnen.

Änderungsantrag 1b (Andreas Spranger)\\
§2 (5) Die Autorenschaft veröffentlichter Beiträge ist für die gesamte Nutzerschaft klar zu kennzeichnen.

Änderungsantrag 2a (Antragssteller)\\
§4 Finanzgeschäfte\\
(1) Zahlungen an soziale Medien sind nicht zu tätigen.\\
(2) Ausnahmen können durch das Plenum beschlossen werden.

Änderungsantrag 2b (Andreas Spranger)\\
§4 Finanzielle Transaktionen mit sozialen Medien bedürfen eines entsprechenden positiven Beschlusses durch das Plenum des Studentenrates der TU Dresden.

\nopagebreak
\vspace{1cm}
Version vom 16.~März~2015.
%Inkraftgetreten am 24.~April~2014.
\\



\normalsize
~\\*[4cm]
\begin{center}
\hspace*{\fill}
%\parbox{7cm}{Matthias Funke\\GF Finanzen}
%\hfill\parbox{7cm}{Andreas Spranger\\GF Hochschulpolitik}
\hspace*{\fill}
\end{center}
