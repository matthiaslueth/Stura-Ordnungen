%Header Wahlordnung multicols
\markright{Wahlordnung}
\setcounter{section}{0} % ist nötig um den Paragrafenzähler zurücksetzen
\begin{multicols}{2}

\section*{Vorbemerkung}
\Satz Aufgrund von § 26 Abs. 1 des Gesetzes über die Hochschulen im Freistaat Sachsen (Sächsisches Hochschulgesetz – SächsHSG) erlässt der Studentenrat der Studierendenschaft der Technischen Universität Dresden folgende Wahlordnung. Der in dieser Ordnung verwendete Begriff „Studierendenschaft“ entspricht der Studentenschaft im Sinne des § 25 SächsHSG.


%\begin{description}
%\item[1. Abschnitt] Grundsätze der Studentenschaft
%\item[2. Abschnitt] Fachschaften
%\item[3. Abschnitt] Studentenrat
%\item[4. Abschnitt] Legislative des StuRa
%\item[5. Abschnitt] Exekutive des StuRa
%\item[6. Abschnitt] Schlussbestimmungen
%\end{description}

\section*{Erster Abschnitt}

\section{Geltungsbereich und Mandatsdauer}

\Abs \Satz Diese Wahlordnung gilt für:
\begin{enumerate}
\item die Wahlen zu den Fachschaftsräten
\item die Wahlen zum Studentenrat
\end{enumerate}

\Abs \Satz Die Mitglieder des Studentenrates und der Fachschaftsräte werden für ein Jahr gewählt und bleiben bis zur Konstituierung des neuen Fachschafts- beziehungsweise Studentenrats im Amt.

\section*{Zweiter Abschnitt - Die Fachschaftsräte}
\section  {Wahlgrundsätze}
\Abs \Satz Die Wahlen sind nach den Grundsätzen des § 26 Abs. 1 SächsHSG (frei, gleich, geheim) durchzuführen.
\Abs \Satz Die Wahl muss barrierefrei gestaltet werden.

\section {Wahlorgane, Zusammensetzung und Aufgaben}
\Abs \Satz Wahlorgane sind der Wahlausschuss, die Wahlleiterin und die Abstimmungsausschüsse (§ 11 Absatz 2)\. Die Wahlbewerber dürfen weder Mitglied im Wahlausschuss noch im Abstimmungsausschuss der eigenen Fachschaft sein\. Eine gleichzeitige Mitgliedschaft in mehreren Wahlorganen ist unzulässig\. Dies betrifft nicht die gleichzeitige Mitgliedschaft des Wahlleiters im Wahlausschuss.

\Abs \Satz Der Wahlausschuss besteht auf fünf bis sieben Mitgliedern\. Die Mitglieder des Wahlausschusses werden vom Studentenrat bestellt\. Sie müssen wahlberechtigt im Sinne von § 4 Abs. 1 sein\. Diese Bestellung erfolgt so rechtzeitig, dass der Wahlausschuss und die Wahlleiterin ihre Aufgaben innerhalb der vorgeschriebenen Fristen erfüllen können\. Die Zusammensetzung des Wahlausschusses wird mit dem Protokoll des Studentenrates veröffentlicht\. Die Amtszeit des Wahlausschusses dauert bis zur erneuten Bestellung eines Wahlausschusses an\. Sie beträgt in der Regel ein Jahr.

\Abs \Satz Der Wahlausschuss nimmt die ihm durch diese Wahlordnung übertragenen Aufgaben wahr\. Er beschließt über die Regelungen von Einzelheiten der Wahlvorbereitungen und der Wahldurchführung, insbesondere über den Wahltermin.

\Abs \Satz Der Wahlausschuss wählt aus seiner Mitte die Wahlleiterin und ihre Stellvertreterin\. Bei Stimmengleichheit entscheidet das Los\. Die erste Sitzung des Wahlausschusses wird vom Geschäftsführer Finanzen des Studentenrates einberufen und von diesem bis zur Wahl der Wahlleiterin geleitet.

\Abs \Satz Die Wahlleiterin ist für die ordnungsgemäße Vorbereitung und Durchführung der Wahl verantwortlich\. Sie sorgt insbesondere für: 
\begin{enumerate}
\item die Bekanntgabe der Wahlausschreibung
\item die Erstellung des Wählerverzeichnisses
\item den Druck der Stimmzettel sowie die Bereitstellung der Wahleinrichtungen
\end{enumerate}
\Satz Sie führt die Beschlüsse des Wahlausschusses aus.

\Abs \Satz Die Sitzungen des Wahlausschusses sollen vom Wahlleiter geleitet werden und können von jedem Mitglied einberufen werden\. Der Wahlausschuss ist beschlussfähig, wenn mehr als die Hälfte der Mitglieder anwesend sind und die Sitzung ordnungsgemäß einberufen wurde\. Der Wahlausschuss entscheidet mit der Mehrheit der Stimmen der Anwesenden\. Kann in einer Angelegenheit eine Entscheidung des Wahlausschusses nicht rechtzeitig herbeigeführt werden, so entscheidet der Wahlleiter\. Von dieser Entscheidung ist der Wahlausschuss unverzüglich zu unterrichten.

\Abs \Satz Die Wahlorgane haben bei ihren Entscheidungen zu berücksichtigen, dass durch die Regelung des Wahlverfahrens und die Bestimmung des Zeitpunktes der Wahl die Voraussetzungen für eine möglichst hohe Wahlbeteiligung geschaffen werden.

\Abs \Satz Die Wahlorgane können zur Erfüllung ihrer Aufgaben Wahlhelferinnen heranziehen.

\Abs \Satz Die Wahlorgane und die Wahlhelferinnen sind zur unparteiischen und gewissenhaften Erfüllung ihrer Aufgaben verpflichtet\. Sie üben ihre Tätigkeit ehrenamtlich aus.

\section{Wahlberechtigung und Wählbarkeit}
\Abs \Satz Wahlberechtigt (aktives Wahlrecht) und wählbar (passives Wahlrecht) ist jedes Mitglied der Studierendenschaft nach § 24 Abs. 1 SächsHSG\. Gasthörerinnen besitzen kein Wahlrecht.
\newpage %Formatkorrektur, mal gucken ob wir da was besseres finden
\Abs \Satz Mitglieder der Studierendenschaft, die mehr als einer Fachschaft angehören, geben bis zur Schließung des Wählerverzeichnisses ab, in welcher Fachschaft sie ihr Wahlrecht ausüben\. Wird diese Erklärung nicht abgegeben, bestimmt sich die Wahlberechtigung nach jener Fachschaft, die für den ersten Eintrag auf dem Studentenausweis zugeordnet ist.

\Abs \Satz Mit dem Verlust des aktiven Wahlrechts entfällt auch das entsprechende passive Wahlrecht\. Die Betroffene scheidet als Mitglied aus dem Fachschaftsrat aus.

\section{Ausübung des Wahlrechts, Wählerverzeichnis}
\Abs \Satz Das aktive und passive Wahlrecht für die Wahlen nach § 1 Abs. 1 Nr. 1 können nur Wahlberechtigte ausüben, die in das Wählerverzeichnis eingetragen sind.

\Abs \Satz Das Wählerverzeichnis wird von der zentralen Universitätsverwaltung erstellt\. Die Wahlleiterin nach dieser Ordnung setzt den Kanzler der TU Dresden mit einer Vorlaufzeit von mindestens 14 Tagen über die beabsichtigte Abforderung des Wählerverzeichnisses in Kenntnis\. Das Wählerverzeichnis gliedert sich nach Fachschaften\. Im Übrigen ist es in alphabetischer Reihenfolge zu führen oder in anderer Weise übersichtlich zu gestalten\. Es muss den Namen, den Vornamen, das Geburtsdatum und das Geschlecht der Wahlberechtigten sowie ein Feld für Bemerkungen enthalten\. Rechtzeitig vor der Auslegung nach § 3 Satz 2 ist ein den Anforderungen dieser Wahlordnung entsprechender Ausdruck zu erstellen.

\Abs \Satz Am 14. Tag vor dem ersten Wahltag wird das Wählerverzeichnis geschlossen\. Es wird während der letzten sieben Arbeitstage vor der Schließung zur Einsicht ausgelegt\. Arbeitstage im Sinne dieser Ordnung sind Wochentage Montag bis Freitag mit Ausnahme der gesetzlichen Feiertage.

\Abs \Satz Gegen die Nichteintragung oder eine falsche Eintragung in ein Wählerverzeichnis kann jede Wahlberechtigte schriftlich während der Dauer der Auslegung Erinnerung bei der Wahlleiterin einlegen\. Die Wahlleiterin trifft unverzüglich, spätestens innerhalb von 3 Kalendertagen nach Schließung des Wählerverzeichnisses eine Entscheidung\. Die betroffene Person soll vorher gehört werden\. Ist die Erinnerung begründet, so berichtigt die Wahlleiterin das Wählerverzeichnis.

\Abs \Satz Eine Berichtigung hinsichtlich der in Abs. 2 Satz 4 bis 6 genannten Angaben ist von der Wahlleiterin auch nach Schließung des Wählerverzeichnisses von Amts wegen vorzunehmen. \\
\Satz Die Wahlleiterin hat auch dann eine Berichtigung des Wählerverzeichnisses vorzunehmen, wenn ihr bis zum Wahltag Tatsachen bekannt werden, die zu einem Verlust der Wahlberechtigung bzw. Wählbarkeit am Wahltag führen (z.B. Ausscheiden aus der Studierendenschaft)\. Eine Berichtigung des Wählerverzeichnisses nach dessen Schließung ist durch die Wahlleiterin in einer Anlage zum Wählerverzeichnis zu vermerken.

\section{Wahlausschreibung}
\Abs \Satz Spätestens am 28. Kalendertag vor dem ersten Wahltag erlässt die Wahlleiterin die Wahlausschreibung\. Sie wird auf den Internetseiten des Studentenrats und durch Aushang bekannt gemacht.

\Abs \Satz Die Wahlausschreibung muss folgende Punkte enthalten:
\begin{enumerate}
\item den Ort und Tag ihres Erlasses,
\item die Erklärung, dass die Vertreter der Fachschaften gewählt werden sollen,
\item den Hinweis, wer wahlberechtigt ist,
\item die Zahl der zu stellenden Vertreter,
\item die Angabe, wann und wo das Wählerverzeichnis zur Einsicht ausliegt,
\item den Hinweis, dass die Ausübung des Wahlrechtes von der Eintragung in das Wählerverzeichnis abhängt, sowie den Hinweis auf die Fristen nach § 5 Abs. 4 und 5,
\item die Aufforderung, Wahlvorschläge einzureichen, den Zeitraum für die Abgabe der Wahlvorschläge und den letzten Tag der Einreichungsfrist,
\item den Hinweis, dass nur fristgerecht eingereichte Wahlvorschläge berücksichtigt werden und dass nur gewählt werden kann, wer zur Wahl vorgeschlagen wurde,
\item den Ort, an dem die Wahlvorschläge bekannt gemacht werden,
\item den Wahltermin, den Ort und die Zeit der jeweiligen Stimmabgabe,
\item den Hinweis, dass die Möglichkeit der Briefwahl besteht;\\ zur Erläuterung ist ein Hinweis auf § 12 dieser Wahlordnung ausreichend,
\item den Hinweis darauf, dass die Wahlberechtigten keine Wahlbenachrichtigung erhalten.
\end{enumerate}

\section{Wahltermine, Zeit und Ort der Stimmabgabe}
\Abs \Satz Die Wahlen finden in der Vorlesungszeit so rechtzeitig statt, dass die konstituierenden Sitzungen der Fachschaftsräte und des Studentenrates vor dem Ende der Vorlesungszeit desselben Semesters durchgeführt werden können\. Sie sollen in der Regel im Wintersemester stattfinden.

\Abs \Satz Die Stimmabgabe ist an drei aufeinander folgenden nicht vorlesungsfreien Tagen durchzuführen\. Die Zeiten der Stimmabgabe werden vom Wahlausschuss bestimmt.

\Abs \Satz Die Wahlen finden für alle Fachschaften an den gleichen Tagen statt, die Uhrzeiten für die Stimmabgabe müssen nicht für alle Fachschaften gleich sein\. Ein Wechsel des Abstimmungsraumes innerhalb eines Abstimmungstages ist möglich\. Der Wahlausschuss stellt sicher, dass bei Wechsel des Abstimmungsraumes ein Zeitintervall von einer Stunde eingehalten wird\. Die vom Wahlausschuss beschlossenen und veröffentlichten Orte sind zwingend einzuhalten\. Die Abstimmungsräume müssen barrierefrei zugänglich sein

\section{Wahlvorschläge}
\Abs \Satz Wahlvorschläge sind nur als Einzelwahlvorschläge zulässig.

\Abs \Satz Wahlvorschläge bedürfen der Schriftform, zulässig sind auch mehrere Einzelwahlvorschläge auf einem Dokument in Tabellenform\. Aus den Wahlvorschlägen muss ersichtlich sein, dass sie die Wahl gemäß § 1 Abs. 1 Nr. 1 (Fachschaftsräte) betreffen\. Es muss weiterhin ersichtlich sein, welche Fachschaft sie betreffen\. Ein Wahlvorschlag muss den Namen, den Vornamen, den Studiengang und das Fachsemester, das Geburtsdatum, das Geschlecht sowie eine E-Mailadresse der Bewerberin enthalten.

\Abs \Satz Die Bewerberin hat auf dem Wahlvorschlag ihr Einverständnis schriftlich zu erklären oder eine Entsprechende schriftliche Erklärung gesondert abzugeben\. Mit diesem Einverständnis soll auch das Einverständnis darüber verbunden werden, dass Mitteilungen und Erklärungen der Wahlorgane gegenüber der Bewerberin in Textform (E-Mail) erfolgen können.

\Abs \Satz Eine Bewerberin darf nur für eine Fachschaft kandidieren.

\Abs \Satz Vorgeschlagene Bewerberinnen können durch schriftliche Erklärung gegenüber dem Wahlleiter ihre Bewerbung zurücknehmen, solange nicht über die Zulassung des Wahlvorschlags entschieden ist.

\Abs \Satz Wahlvorschläge können nur innerhalb der vom Wahlleiter festgesetzten Frist eingereicht werden\. Diese Frist beträgt zwei Wochen und endet regelmäßig am 14. Kalendertag vor dem ersten Wahltag.

\Abs \Satz Werbung für einen Wahlvorschlag (Wahlkampf) ist ab dem Tage der Einreichung des Wahlvorschlages zulässig.

\section{Prüfung der Wahlvorschläge}
\Abs \Satz Der Wahlausschuss prüft die Wahlvorschläge unverzüglich nach ihrem Eingang und entscheidet über ihre Gültigkeit und Zulassung\. Stellt er Mängel fest, gibt er den Wahlvorschlag an die Bewerberin mit der Aufforderung zurück, die Mängel innerhalb einer Frist von drei Kalendertagen zu beseitigen\. Werden die Mängel nicht fristgerecht beseitigt, ist der Wahlvorschlag ungültig.

\Abs \Satz Aufgrund der zugelassenen Wahlvorschläge werden vom Wahlleiter Stimmzettel erstellt\. Die Reihenfolge der Wahlvorschläge auf dem Stimmzettel wird durch den Wahlausschuss per Los bestimmt.

\Abs \Satz Spätestens am 11 Kalendertag vor dem ersten Wahltag gibt der Wahlleiter die zugelassenen Wahlvorschläge bekannt\. Mit der Bekanntgabe ist die weitere Werbung für nicht zugelassene Wahlvorschläge unzulässig.

\section{Vorbereitung der Wahl und Gestaltung der Wahlunterlagen}
\Abs \Satz Für die Wahl jedes Fachschaftsrates werden gesonderte Stimmzettel hergestellt\. Auf den Stimmzetteln sind die Wahlvorschläge jeweils in Reihenfolge der Losnummern mit den in § 8 Abs. 2 genannten Angaben aufzuführen, jedoch ohne die Angabe zu Geburtsdatum, Geschlecht und E-Mailadresse\. Auf den Stimmzetteln ist auf die Möglichkeit der Stimmabgabe nach § 11 Abs. 4 hinzuweisen\. Die Stimmzettel sind nach den Grundsätzen der Barrierefreiheit anzufertigen.

\Abs \Satz Im Übrigen entscheidet der Wahlausschuss über die äußere Gestaltung der Wahlunterlagen.

\section {Stimmabgabe}
\Abs \Satz Für jeden Abstimmungsraum wird von der Wahlleiterin ein Abstimmungsausschuss bestellt, der so groß sein soll, dass die Einhaltung von §7(4) gewährleistet ist\. Er muss mindestens aus drei Personen bestehen\. Zur Vorbereitung der Bestellung schlägt der amtierende Fachschaftsrat bis zum 21. Tag vor dem ersten Abstimmungstag eine Vorsitzende vor\. Sobald diese durch die Wahlleiterin ernannt wird, schlägt sie der Wahlleiterin mindestens zwei weitere Mitglieder vor\. Mindestens zwei Mitglieder des Abstimmungsausschusses müssen ständig im Abstimmungsraum anwesend sein, solange dieser für die Stimmabgabe geöffnet ist\. Jegliche Beeinflussung der Wahlberechtigten im Abstimmungsraum ist unzulässig\. Jedes Mitglied des Abstimmungsausschusses kann im näheren Umkreis von Wahllokalen sichtliche Beeinflussung von Wahlbeteiligten sowie den Aufenthalt von Personen untersagen die dort nicht aus dienstlichen Gründen oder zur Wahlhandlung anwesend sein müssen\. Dieser Umkreis ist zu kennzeichnen.

\Abs \Satz Die Wahlleiterin trifft Vorkehrungen, dass der Wähler den Stimmzettel in dem ihm gemäß § 7 zugewiesenen Abstimmungsraum unbeobachtet kennzeichnen kann\. Für die Aufnahme der Stimmzettel sind Wahlurnen zu verwenden\. Vor der ersten Stimmabgabe hat der Abstimmungsausschuss sicherzustellen, dass die Urne leer ist.

\Abs \Satz Die Stimmberechtigten erhalten vom Wahlvorstand beim Betreten des Abstimmungsraumes die erforderlichen Stimmzettel, sofern sie im jeweiligen Abstimmungsraum wahlberechtigt sind und noch nicht gewählt haben\. Eine Vertretung bei der Stimmabgabe ist unzulässig.

\Abs \Satz Der Wähler gibt seine Stimme ab, indem er eindeutig kenntlich macht, welche Kandidaten er wählt\. Bei jeder Wahl kann der Wahlberechtigte bis zu drei Stimmen abgeben\. Die Wählerin kann einem Bewerberin bis zu drei Stimmen geben (kumulieren) oder auch ihre drei Stimmen auf mehrere Bewerberin verteilen (panaschieren).

\Abs \Satz Vor Einwurf des gefalteten Stimmzettels in die Urne ist die Wahlberechtigung anhand des Wählerverzeichnisses zu überprüfen\. Die Wählerin hat sich auf Verlangen über seine Person auszuweisen\. Unmittelbar danach wirft sie ihren Stimmzettel in die Wahlurne\. Die Stimmabgabe ist im Wählerverzeichnis zu vermerken.

\Abs \Satz Wird die Wahlhandlung unterbrochen oder wird das Wahlergebnis nicht unmittelbar nach Abschluss der Stimmabgabe festgestellt, hat der Abstimmungsausschuss für die Zwischenzeit die Wahlurne zu verschließen und aufzubewahren\. Er hat sicherzustellen, das der Einwurf oder die Entnahme von Stimmzetteln ohne Beschädigung des Verschlusses unmöglich sind\. Bei erneuter Öffnung der Wahlurne oder bei Entnahme der Stimmzettel hat sich der Abstimmungsausschuss davon zu überzeugen, dass der Verschluss unversehrt geblieben ist.

\Abs \Satz Nach Ablauf der für die Stimmabgabe festgesetzten Zeit dürfen nur noch die Wahlberechtigten ihre Stimme abgeben, die sich zu diesem Zeitpunkt im Wahlraum aufhalten\. Nachdem diese ihre Stimmzettel in die Wahlurne eingeworfen haben und im Wählerverzeichnis vermerkt worden sind, erklärt der Abstimmungsausschuss am letzten Tag die Stimmabgabe für beendet.

\section{Briefwahl}
\Abs \Satz Die Stimmabgabe ist auch in der Form der Briefwahl zulässig.

\Abs \Satz Eine Wahlberechtigte, die eine Stimmabgabe in der Form der Briefwahl beabsichtigt, beantragt bei der Wahlleiterin schriftlich die Übersendung oder Aushändigung der Wahlunterlagen\. Der eigenhändig unterzeichnete Antrag muss:
\begin{itemize}
\item [a.] beim Antrag auf Übersendung spätestens am 14 Kalendertag
\item [b.] beim Antrag auf Aushändigung spätestens am 5. Kalendertag
\end {itemize}
vor dem ersten Wahltag bei der Wahlleiterin eingehen\. Die Wahlleiterin prüft die Wahlberechtigung. Sie sendet der Wahlberechtigten unverzüglich nach Bekanntgabe der zugelassenen Wahlvorschläge die Wahlunterlagen zu oder händigt sie aus\. Sie vermerkt die Übersendung oder Aushändigung im Wählerverzeichnis\. Ein Wahlberechtigter, bei dem im Wählerverzeichnis die Übersendung oder Aushändigung der Briefwahlunterlagen vermerkt ist, kann seine Stimme nur durch Briefwahl abgeben.

\Abs \Satz Die Wahlunterlagen bestehen aus einem Stimmzettel, einem amtlich gekennzeichneten Wahlumschlag, einem Wahlschein und einem für das Inland freigemachten Briefwahlumschlag, der die Anschrift des Wahlleiters und als Absender den Namen und die Anschrift der wahlberechtigten Person sowie den Vermerk \glqq schriftliche Stimmabgabe\grqq\, trägt\. Der Wahlschein enthält mindestens den Namen, Vornamen, die Anschrift sowie die vorgedruckte Erklärung, den beigefügten Stimmzettel persönlich gekennzeichnet zu haben.

\Abs \Satz Beim Antrag auf Aushändigung erfolgt diese im Servicebüro des Studentenrat.

\Abs \Satz Die Stimmabgabe erfolgt dadurch dass: 1. die Briefwählerin den Stimmzettel persönlich gemäß § 11 Absatz 4 kennzeichnet, in den Wahlumschlag legt, und diesen verschließt, 2. sie den Wahlschein mit der vorgedruckten Erklärung persönlich unterzeichnet, 3. sie den Wahlschein und den Wahlumschlag in den zugegangenen Briefwahlumschlag legt und diesen verschließt (Wahlbrief) und 4. der Wahlbrief rechtzeitig vor Ablauf der für die Stimmabgabe festgesetzten Frist dem Wahlleiter zugeht. % in einer Änderung könnte, man das mal untereinander schreiben

\Abs \Satz Auf dem Wahlbrief sind von der Wahlleiterin oder einer von ihr benannten Wahlhelferin Tag und Uhrzeit des Eingangs zu vermerken\. Die eingegangenen Wahlbriefe werden gezählt und ihre Anzahl in die Wahlniederschrift nach § 15 eingetragen.

\Abs \Satz Spätestens Nach Ablauf der für die Stimmabgabe festgesetzten Zeit werden zur Überprüfung die rechtzeitig eingegangenen Wahlbriefe geöffnet; die nicht rechtzeitig im Sinne von Absatz 5 eingegangenen Wahlbriefe bleiben ungeöffnet\. Die Wahlscheine werden mit den Eintragungen im Wählerverzeichnis verglichen.\\
\Satz Ein Wahlbrief wird zurückgewiesen, wenn
\begin{enumerate}
\item er nicht bis zum Ablauf der für die Stimmabgabe festgesetzten Zeit eingegangen ist,
\item er unverschlossen eingegangen ist,
\item der Wahlumschlag nicht amtlich gekennzeichnet oder mit einem Kennzeichen
versehen ist,
\item der Wahlumschlag kein mit der unterschriebenen vorgedruckten Erklärung
versehener Wahlschein beigefügt ist,
\item sich Stimmzettel außerhalb des Wahlumschlags befinden oder
\item die Angaben auf dem Wahlschein mit den Eintragungen im Wählerverzeichnis
nicht übereinstimmen und keine Berichtigung nach § 5 Abs. 6 erfolgt.
\end{enumerate}

\Abs \Satz In den Fällen des Absatz 7 Satz 3 liegt eine Stimmabgabe nicht vor\. Die zurückgewiesenen Wahlbriefe sind einschließlich ihres Inhaltes auszusondern und im Fall des Absatz 7 Satz 3 Nr. 1 ungeöffnet, im Übrigen ohne Öffnung des Wahlumschlags, der Wahlniederschrift nach § 15 als Anlage beizufügen.

\Abs \Satz Die Wahlumschläge aus nicht zurückgewiesenen Wahlbriefen werden nach der im Wählerverzeichnis vermerkten Stimmabgabe ungeöffnet in die Wahlurne gelegt.

\section{Auszählung}
\Abs \Satz Unverzüglich nach Beendigung der Stimmabgabe (§ 11 Abs. 7) sind von den Abstimmungsausschüssen die Abstimmungsergebnisse vorläufig zu ermitteln und dem Wahlausschuss zusammen mit den Wahlunterlagen zu übergeben\. Die Bildung von Zählgruppen, die mindestens aus einem Mitglied des Abstimmungsausschusses und einer Hilfskraft bestehen müssen ist zulässig\. Spätestens 6 Tage nach Beendigung der Stimmabgabe zählt der Wahlausschuss in Zweifelsfällen nach\. Die Auszählung ist hochschulöffentlich.

\Abs \Satz Sofort nach der Öffnung der Wahlurnen werden die ungeöffneten Wahlbriefe geöffnet und unter Wahrung des Wahlgeheimnisses deren Inhalt unter die übrigen Stimmzettel gemischt\. Dann werden die Stimmzettel auf ihre Gültigkeit überprüft\. Ein abgegebener Stimmzettel ist ungültig,
\begin{enumerate}
\item wenn keine Bewerberin gekennzeichnet wurde,
\item wenn er nicht als amtlich erkennbar ist,
\item wenn der Stimmzettel einen Zusatz, der nicht der Kennzeichnung der gewählten
Bewerberin dient oder einen Vorbehalt enthält,
\item wenn mehr als drei Stimmen abgegeben wurden,
\item wenn aus dem Stimmzettel der Wille der Wählerin nicht zweifelsfrei erkennbar ist
\end {enumerate}

\Abs \Satz Bei Zweifeln über die Gültigkeit oder Ungültigkeit der Stimmabgabe entscheidet der Wahlausschuss.

\Abs \Satz Der Wahlausschuss stellt für jede Wahl die Zahl der abgegebenen Stimmzettel, die Zahl der ungültigen Stimmzettel sowie die Zahlen der gültigen Stimmen fest, die auf die einzelnen Wahlvorschläge und Bewerberinnen entfallen sind\. Die Zahl der abgegebenen Stimmzettel muss mit der Zahl der Abstimmungsvermerke im Wählerverzeichnis übereinstimmen\. Ergibt sich auch nach wiederholter Zählung keine Übereinstimmung, so ist dies in der Niederschrift anzugeben und, soweit möglich, zu erläutern.

\section {Feststellung des Wahlergebnisses}
\Abs \Satz Der Wahlausschuss hat die von den Abstimmungsausschüssen getroffenen Entscheidungen über die Gültigkeit von Stimmzetteln und Stimmen zu überprüfen und gegebenenfalls das Ergebnis der Zählung zu berichtigen. Er stellt die Ergebnisse fest. Er stellt weiter die Gesamtzahl der abgegebenen Stimmen, die Zahl der ungültigen Stimmen und die Anzahl der gültigen Stimmen je Bewerberin und die damit gewählten Bewerberinnen und die Reihenfolge der Ersatzvertreter nach Maßgabe der Absätze 3 bis 5 fest.

\Abs \Satz Die Wahlleiterin gibt das festgestellte Wahlergebnis spätestens sieben Arbeitstage nach Abschluss der Wahl auf den Internetseiten des Studentenrats bekannt. Sie hat es von Amts wegen zu berichtigen, wenn innerhalb von vier Monaten nach Feststellung Schreibfehler, Rechenfehler oder ähnliche Unrichtigkeiten bekannt werden.

\Abs \Satz Zunächst werden die dem Geschlecht in der Minderheit zustehenden Mindestsitze verteilt\. Dazu werden die dem Geschlecht in der Minderheit zustehenden Mindestsitze mit Angehörigen dieses Geschlechts in der Reihenfolge der jeweils höchsten auf sie entfallenden Stimmenzahlen besetzt, sofern diese mindestens eine Stimme erhalten haben.

\setcounter{sentence}{0} %Tut mir Leid, ich hatte keine Ahnung, wie ich 3b sinnvoll einfügen sollte
(3b)\Satz Ist kein Geschlecht in einer Fachschaft mit weniger als 40\% vertreten, so findet Abs. 3 Satz 1 keine Anwendung\. Stattdessen werden dann zunächst jeweils je Geschlecht abgerundete 40\% der Sitze in der Reihenfolge der jeweils höchsten auf die Bewerberinnen entfallenden Stimmen besetzt, sofern sie Mindestens eine Stimme erhalten haben.

\Abs \Satz Maßgeblich für die Bestimmung des Geschlechtes in der Minderheit und die Anzahl der Mindestsitze einer Fachschaft ist das Wählerverzeichnis\. Die Anzahl der Mindestsitze ergibt sich aus dem aufgerundeten Anteil des Minderheitengeschlechts im Verhältnis zu der Zahl der Sitze im jeweiligen Fachschaftsrat\. Sollte es für die nach Satz 2 vorgesehenen Sitze nicht genügend Bewerberinnen des Minderheitengeschlechts geben, entfallen die restlichen Sitze jeweils auf das andere Geschlecht.

\Abs \Satz Nach der Verteilung der Mindestsitze des Geschlechts in der Minderheit nach Absatz 3 bzw. nach der Verteilung der Sitze je Geschlecht nach Absatz 3b erfolgt die Verteilung der weiteren Sitze\. Die weiteren Sitze werden mit Bewerberinnen und Bewerbern, unabhängig von ihrem Geschlecht, in der Reihenfolge der jeweils höchsten auf sie entfallenden Stimmenzahlen besetzt.

\Abs \Satz Entfällt auf mehrere Bewerberinnen die gleiche Stimmenanzahl, so entscheidet der Wahlausschuss in einem zu protokollierenden Verfahren durch das Los über die Reihung der Kandidaten. Zuvor sind die strittigen Stimmen erneut auszuzählen\. Auf das Verfahren nach Satz 1 und 2 kann verzichtet werden, wenn alle betreffenden Kandidaten einen Sitz im Fachschaftsrat erhalten\. Die Entscheidung des Loses ist nicht anfechtbar.

\Abs \Satz Gibt es mehrere Bewerber mit mindestens einer Stimme als Sitze vorhanden sind, so sind die nicht gewählten Bewerber in absteigender Reihenfolge ihrer Stimmanzahl Ersatzvertreter in der nach Absatz (4) vorgesehenen Aufteilung.

\section {Wahlniederschrift, Aufbewahrung von Wahlunterlagen}
\Abs \Satz Über die Verhandlung des Wahlausschusses und seine Beschlüsse sowie über die Wahlhandlungen und die Tätigkeit der Wahlorgane sind Niederschriften zu fertigen\. Die Niederschriften über die Tätigkeit der Wahlorgane werden von den Mitgliedern des jeweiligen Wahlorgans, die übrigen von der Wahlleiterin unterzeichnet.

\Abs \Satz Die Wahlniederschriften sollen insbesondere den Gang der Wahlhandlung aufzeichnen, das Wahlergebnis festhalten und besondere Vorkommnisse vermerken.

\Abs \Satz Die Wählerverzeichnisse, Stimmzettel und Wahlniederschriften sind bis zum Ablauf der Amtszeit der gewählten VertreterInnen aufzubewahren.

\section{Annahme der Wahl}
\Abs \Satz Die Wahlleiterin hat die Gewählten unverzüglich von ihrer Wahl schriftlich zu verständigen. Die Wahl gilt als angenommen, wenn nicht spätestens am fünften Tag nach Zugang der Benachrichtigung der Wahlleiterin eine Ablehnung der Wahl in schriftlicher Form aus wichtigem Grund vorliegt. Ob ein wichtiger Grund vorliegt entscheidet der Wahlausschuss.

\Abs \Satz Nach Annahme der Wahl können die Gewählten von ihrem Amt nur zurücktreten, wenn der Ausübung des Amtes wichtige Gründe entgegenstehen\. Über die Annahme des Rücktritts entscheidet die Wahlleiterin.

\section {Nachrücken von Ersatzvertretern}
\Abs \Satz Wird die Wahl von einer Person rechtswirksam nicht angenommen, rückt der Ersatzvertreter nach, der gemäß § 14 in der Reihenfolge der Ersatzvertreter der Nächste ist\. Sind Ersatzvertreter nicht vorhanden, verringert sich die Zahl der Sitze des jeweiligen Fachschaftsrates entsprechend.

\Abs \Satz Scheidet eine gewählte Vertreterin aus, gilt Absatz 1 und § 16 entsprechend.

\section{Wahlprüfung}
\Abs \Satz Jede Wahlberechtigte kann nach der Bekanntgabe des Wahlergebnisses die Wahl innerhalb von 6 Kalendertagen unter Angabe von Gründen anfechten. Die Anfechtung erfolgt durch schriftliche Erklärung gegenüber der Wahlleiterin.

\Abs \Satz Die Anfechtung ist begründet, wenn wesentliche Vorschriften über das Wahlrecht, die Wählbarkeit oder das Wahlverfahren verletzt worden sind und diese Verletzung zu einer fehlerhaften Sitzverteilung geführt hat oder hätte führen können. Eine Anfechtung der Wahl mit der Begründung, das eine Wahlberechtigte an der Ausübung ihres Wahlrechtes gehindert gewesen sei, weil sie nicht oder nicht richtig in das Wählerverzeichnis eingetragen worden sei oder das eine Person an der Wahl teilgenommen habe, die zwar in das Wählerverzeichnis eingetragen, aber nicht wahlberechtigt gewesen sei, ist nicht zulässig.

\Abs \Satz Über die Anfechtung entscheidet der Wahlausschuss. Der Beschluss ist schriftlich zu begründen, mit einer Rechtsbehelfsbelehrung zu versehen und der Antragstellenden sowie der unmittelbar betroffenen Person zuzustellen. Ist die Anfechtung begründet, hat der Wahlausschuss entweder das Wahlergebnis bei fehlerhafter Auszählung zu berichtigen oder die Wahl in dem erforderlichen Umfang für ungültig zu erklären und insoweit eine Wiederholungswahl anzuordnen. Vorbehaltlich einer anderweitigen Entscheidung im Wahlprüfungsverfahren wird bei der Wiederholungswahl nach den gleichen Vorschlägen und aufgrund des gleichen Wählerverzeichnisses gewählt wie bei der für ungültig erklärten Wahl; Wirkt sich ein Verstoß über die Sitzverteilung nur in einer Fachschaft aus, ist nur diese Wahl für ungültig zu erklären und zu wiederholen. Eine Wiederholung der Wahl ist unverzüglich durchzuführen. Die Wahlleiterin legt den Wahltermin und die Zeit der Stimmabgabe fest.

\section{Fristen}
\Abs \Satz Soweit für die Stellung von Anträgen oder die Einreichung von Vorschlägen die Wahrung einer Frist vorgeschrieben ist, läuft die Frist am letzten Tag um 16 Uhr ab. § 12 Abs. 5 Nr. 4 bleibt unberührt.

\section{Konstituierung der Fachschaftsräte}
\Abs \Satz Die Fachschaftsräte konstituieren sich frühestens 7 und spätestens 21 Kalendertage nach der Bekanntgabe der Wahlergebnisse.

\section*{Dritter Abschnitt - Der Studentenrat}
\section{Wahl des Studentenrats}
\Abs \Satz Der Studentenrat setzt sich aus den von den einzelnen Fachschaftsräten entsandten Vertretern zusammen.

\Abs \Satz Der Studentenrat hat maximal 39 Sitze, die wie folgt besetzt werden:
\begin{enumerate}
\item Jeder Fachschaftsrat entsendet durch Wahl einen Vertreter (Basisvertreter)
\setcounter{sentence}{0}
\item \Satz Entsprechend der Größe der jeweiligen Fachschaft können zusätzlich bis zu drei Vertreter nach folgendem Verfahren entsandt werden\. Es werden pro Fachschaft drei Kennzahlen durch Multiplikation der Anzahl der Fachaftsmitglieder mit 30, 17, 7 und anschließender Division durch die Anzahl der Mitglieder der Studierendenschaft gebildet\. Anhand der Kennzahlen größer eins werden nach dem Höchstzahlverfahren die weiteren Vertreter bis zur maximalen Größe des Studentenrates von 33 Basis- und weiteren Vertretern entsandt.
\item Für Fachschaften die mehr als einen Vertreter nach Punkt 1 und 2 entsenden muss jedes Geschlecht mindestens zur abgerundeten Hälfte vertreten sein.
\setcounter{sentence}{0}
\item \Satz Von 3. kann abgewichen werden, sofern sich innerhalb eines Zeitraums von drei Wochen
nach Ausschreibung des Platzes kein Vertreter des entsprechenden Geschlechts zur Wahl
stellt\. Die Ausschreibung ist auch bei besetztem Platz möglich.
\end{enumerate}

\Abs \Satz Geschäftsführer werden zu Vertretern mit besonderem Sitz (besondere Vertreter), wenn der Fachschaftsrat die maximal mögliche Zahl an Basis- und weiteren Vertretern entsandt hat\. Ist der Geschäftsführer Basis- oder weiterer Vertreter, kann der Fachschaftsrat einen Vertreter neu entsenden.

\Abs \Satz Eine Fachschaft darf insgesamt nicht mehr als fünf Vertreter haben.

\Abs \Satz Entsendet ein Fachschaftsrat weniger weitere Vertreter als ihm das nach Abs. 2 Nr. 2 möglich ist, geht die Möglichkeit der Entsendung dieser Vertreter nach zwei aufeinander folgenden Sitzungen an die nach dem Höchstzahlverfahren gemäß Absatz 2 Nr. 2 nachfolgenden Fachschaften über.

\Abs \Satz Nimmt ein Vertreter an zwei aufeinander folgenden Sitzungen unentschuldigt nicht teil, ruht sein Mandat für die Zeit seiner weiteren Abwesenheit\. Ruhende Mandate weiterer Vertreter werden wie Nichtentsendungen nach Abs. 3 behandelt.

\Abs \Satz Nach Rücktritt oder Abwahl eines Geschäftsführers hat der entsprechende Fachschaftsrat alle Vertreter neu zu entsenden.

\Abs \Satz Die Mitgliedschaft eines Vertreters im Studentenrat endet mit dem Ende der Legislatur des Studentenrates. Ferner endet sie durch Rücktritt, Exmatrikulation, Tod oder Rücknahme der Entsendung durch den Fachschaftsrat.

\section{Konstituierung des Studentenrats}
\Abs \Satz Der Studentenrat konstituiert sich spätestens 28 Tag nach der Bekanntgabe der Wahlergebnisse gemäß § 14 Abs. 2.

\section*{Vierter Abschnitt}
\section{Übergangsbestimmungen}
\Abs \Satz Die Regelung des § 7 Absatz 3 Satz 2 tritt erst im Jahr 2010 in Kraft.

\Abs \Satz Alle Fristen der Wahl der Fachschaftsräten richten sich 2009 nach den Fristen der Universitätswahlen.

\Abs \Satz Mit Inkrafttreten dieser Wahlordnung werden sämtliche anders lautenden Regelungen zur Wahl und der darauf folgenden Zusammensetzung der Fachschaftsräte und des Studentenrates der Technischen Universität ungültig.
\end{multicols}
\nopagebreak
\vspace{1cm}
Inkraftgetreten am 13.~August~2009.
\\ 
  

\footnotesize
Geändert am 06.~Januar~2014\\
§~21 Abs.~2 : NEU Listenpunkt 4\\

\normalsize
~\\*[4cm]
\begin{center}
\hspace*{\fill}
\parbox{7cm}{Jessica Rupf\\GF Soziales}
\hfill\parbox{7cm}{Matthias Funke\\GF Finanzen}
\hspace*{\fill}
\end{center}     
