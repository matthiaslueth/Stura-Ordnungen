%\addchap[Härtefallordnung zur Beitragsordnung \S~4 Abs.~1]{Härtefallordnung\\zur Beitragsordnung §~4 Abs.~1}
\markright{Härtefallordnung}
\setcounter{section}{0}
\begin{multicols}{2}



\section{Allgemeines}
\Abs \Satz In besonders schwerwiegenden sozialen und wirtschaftlichen Notlagen kann die Studentenschaft der TU Dresden einzelnen Mitgliedern der Studentenschaft den Studentenschaftsbeitrag, die Kosten des Semestertickets sowie den Semesterbeitrag für das Studentenwerk auf Antrag zurückerstatten. 


\section{Antragsberechtigte}
\Abs \Satz Antragsberechtigt sind alle Mitglieder der Studentenschaft der TU Dresden\. Die Antragstellerin hat in angemessenem Umfang zur Verbesserung ihrer finanziellen Situation beizutragen\. Der Bezug von Unterhaltsleistungen sowie anderen Sozialleistungen hat Vorrang vor der Anerkennung als Härtefall.

\Abs \Satz Befindet sich die Antragstellerin im Zweitstudium, ist eine Rückerstattung nur in begründeten Ausnahmefällen möglich.

\Abs \Satz Studentinnen, die wegen familiärer Verpflichtungen, chronischer Krankheit oder Behinderung beurlaubt worden sind und das Semesterticket nachkaufen und somit freiwillig Studentenschafts- und Semesterticketbeitrag zahlen, können diese zurückerstattet bekommen, wenn für sie die Regelungen dieser Ordnung zutreffen.

\Abs \Satz Die Anzahl der genehmigungsfähigen Anträge ist auf die Hälfte der Regelsemester des jeweiligen Studienganges begrenzt\. Bei einer ungeraden Anzahl an Regelsemestern wird auf das jeweils volle Semester aufgerundet.\\
\Satz Ablehnungen von Härtefallanträgen beeinflussen nicht die Höchstbezugsdauer.\\
\Satz Ausgenommen von §2 Abs 4 Satz sind Studentinnen ohne Arbeitserlaubnis und Studentinnen mit einer chronischen Krankheit bzw. Behinderung, sowie Studentinnen mit familiären Verpflichtungen\. 
Anträge die vor dem Sommersemester 2015 gestellt wurden fallen nicht unter die Höchstbezugsdauer.


\section{Einkommensbegriff}
\Abs \Satz Einkommen im Sinne dieser Ordnung sind alle Einkünfte nach §2 Abs. 1 und 2 EStG (insbesondere Einkommen aus selbständiger und nicht-selbständiger Arbeit), Stipendien, freiwillige Zuwendungen Dritter, alle Unterhaltsansprüche sowie alle staatlichen Sozialleistungen, insbesondere Leistungen nach dem Bundesausbildungsförderungsgesetz (BAföG), Wohngeld und Kindergeld.

\Abs \Satz Nicht zum Einkommen zählen das Elterngeld bis zu einer Höhe von 300 Euro und Mutterschaftsgeld.

\Abs \Satz Zahlungen aus Studienkrediten und sonstigen Darlehen sind zum Einkommen nicht hinzuzurechnen. Davon ausgenommen sind zinsfreie Darlehen nach BAFöG.

\Abs \Satz Die Einkommensgrenze für eine Bewilligung des Antrags setzt sich aus einem Freibetrag pro Person zuzüglich angemessener Mietkosten und Wohnnebenkosten (Wasser, Strom, Heizung) sowie Beiträge zur Kranken- und Pflegeversicherung, wenn diese selbst zu zahlen sind, zusammen.\\
\Satz Der Freibetrag kann semesterweise vom Plenum mit einfacher Mehrheit geändert werden, muss jedoch mindestens 320€ pro Person betragen und auf der Homepage veröffentlicht werden\. Er gilt solange kein neuer Beschluss gefällt wurde. \\
\Satz Die Angemessenheit der Wohn- und Nebenkosten richtet sich nach der ortsüblichen Mietstufe, die beim Wohngeld Anwendung findet. \\
\Satz  Lebt die Antragstellerin mit einer oder mehreren anderen Person/en (insbesondere eigenen Kindern) in einer Haushalts- und Wirtschaftsgemeinschaft so ist deren Einkommen gemeinsam zu berücksichtigen.\\
\Satz Für jede weitere Person erhöht sich die Einkommensgrenze um den aktuell festgelegten Freibetrag.

\Abs \Satz Zahlt die Antragstellerin Unterhalt für ein eigenes Kind, welches sich nicht im Haushalt befindet, erhöht sich die Einkommensgrenze um den Unterhalt für das Kind, maximal jedoch 350 Euro.

\Abs \Satz Leben zwei Antragsteller in einer Lebenspartnerschaft oder Ehe zusammen, sind Einkommen und Freibeträge gemeinsam zu berücksichtigen.

\section{Form und Fristen}
\Abs \Satz Der Antrag ist persönlich und schriftlich bei der Geschäftsführerin Soziales bzw. bei der von der Geschäftsführung bestimmten Verantwortlichen zu stellen.

\Abs \Satz Die Antragsfrist endet einen Monat nach Beginn des Semesters auf das sich der Antrag bezieht\. Als Tag des Antragseingangs gilt der Tag des Eingangs beim Studentenrat der TU Dresden.


\section{Verfahren}
\Abs \Satz Der Antrag ist fristgerecht einzureichen\. Zur Antragstellung soll das zur Verfügung gestellte Formblatt verwendet werden\. Ein verspätet eingegangener Antrag kann berücksichtigt werden, wenn für die Verspätung besondere, nicht durch den Antragsteller zu vertretende Gründe vorliegen\. Zur Wahrung der Frist kann der Antrag vorläufig auch formlos gestellt werden. Das ausgefüllte Formblatt ist in jedem Fall gemeinsam mit den restlichen Unterlagen nachzureichen.

\Abs \Satz Der Antrag muss folgende Unterlagen enthalten: 
\begin{itemize}
\item Angaben zur Person (Antragsformular) 
\item eine Immatrikulationsbescheinigung sowie eine Kopie des Studentenausweises
\item eine schriftliche Darlegung der aktuellen Situation und Notlage sowie der Bemühungen zur Verbesserung der Situation
\item Nachweis Kranken- und Pflegeversicherung
\item Nachweis Miet- und Wohnnebenkosten
\item die Einkommensverhältnisse nach §3 dieser Ordnung unterbrechungsfrei für 3 Monate in Kopie
\item eine Kopie des BAföG-Ablehnungsbescheides. 
\end{itemize}
\Satz Ist offensichtlich, dass die Antragstellerin nicht BAföG-berechtigt ist, kann auf den Ablehnungsbescheid verzichtet werden\. Es muss ein Personaldokument zur Feststellung der Identität vorgelegt werden.

\Abs \Satz Fehlende Unterlagen sind nach Aufforderung nachzureichen\. Werden fehlende Unterlagen innerhalb einer festgesetzten Frist nicht nachgereicht, wird der Antrag abgelehnt.

\Abs \Satz Die Geschäftsführerin Soziales bzw. die von der Geschäftsführung bestimmte Verantwortliche erarbeitet eine Stellungnahme und legt diese sowie den vollständigen Antrag der Geschäftsführung des Studentenrates zur Beschlussfassung vor.


\section{Haushaltsvorbehalt und Rechtsanspruch}
\Abs \Satz Die Rückerstattung wird aus Mitteln der Studentenschaft der TU Dresden geleistet\. Für die Rückerstattung im Sinne dieser Ordnung ist ein eigenständiger Haushaltstitel im Haushalt der Studentenschaft zu führen.

\Abs \Satz Eine Rückerstattung erfolgt unter dem Vorbehalt verfügbarer Mittel im zugeordneten Haushaltstitel.

\Abs \Satz Auf die Rückerstattung des Beitrages besteht kein Rechtsanspruch.

\Abs \Satz Bei Widerspruch ist der Antrag durch die Geschäfstführerin Soziales, wenn von einer beauftragten Verantwortlichen bearbeitet, zu prüfen. Ist der Antrag durch die Geschäftsführerin Soziales bearbeitet worden, ist er von einer anderen Geschäftsführerin zu prüfen\.
Ist ein Antrag nach Widerspruch angenommen worden, kann eine Rückerstattung nur erfolgen, wenn entsprechende Mittel verfügbar sind.


\section{Inkrafttreten und Übergangsbestimmungen}
\Abs \Satz Die Härtefallordnung tritt zum 01.02.2015 in Kraft\. Gleichzeitig tritt die Härtefallordnung vom 01.04.2014 außer Kraft.

\Abs \Satz Diese Härtefallordnung findet erstmals Anwendung für alle Anträge, die für das Sommersemester 2015 gestellt werden, für Anträge aus vorherigen Semestern und Anträge des Wintersemesters 2014/2015 die bereits bearbeitet sind, findet die damals gültige Ordnung Anwendung\. Die Änderung vom 04.06.2015 findet erstmals Anwendung für alle Anträge, die für das Wintersemester 2015/2016 gestellt werden.

\end{multicols}

\nopagebreak
\vspace{1cm}



\footnotesize

Vollständig neu beschlossen am 13.~November~2008\\

Geändert am 01. Oktober 2010 \\
§ 2 Abs. 1 Satz 1 geändert in "`350 Euro"'; \\
§ 2 Abs. 1 Satz 1 geändert in "`Mietkosten (inklusive aller Wohnnebenkosten)"'; \\
§ 3 Abs. 1 Satz1 geändert in "`Einkünfte"'; \\
§ 3 Abs. 3 neu formuliert; \\
§ 3 Abs. 4 Satz 1 geändert in "`Person/en (insbesondere eigenen Kindern)"'; \\
§ 3 Abs. 4 Satz 2 geändert in "`350 Euro"'; \\
§ 3 Abs. 4 NEU; \\
§ 7 neu formuliert; \\

Geändert am 25.10.2013 \\
§ 1 Abs. 1 Satz 1 "`Studentinnen"' geändert in "`Mitgliedern der Studentenschaft"'; \\
§ 2 Abs. 1 Satz 1 "`Studentinnen"' geändert in "`Mitglieder der Studentenschaft"' und Verschiebung der Einkommensgrenze in § 3 Abs. 4 Satz 1; \\
§ 3 Abs. 4 Satz 1 eingefügt aus § 2 Abs. 1 Satz 1 und Änderung der Grenze von 350 Euro auf 370 Euro, Spezifizierung der Nebenkosten, Aufnahme der Krankenversicherung; \\
§ 3 Abs. 6 NEU;
§ 4 Abs. 1 Satz 1 "`Verantwortlichen für Soziales"' geändert zu "`Verantwortlichen"';\\
§ 5 Abs. 1 Satz 4 NEU;\\
§ 5 Abs. 2 Satz 1 vervollständigt;\\
§ 5 Abs. 2 Satz 2 NEU;\\
§ 5 Abs. 4 Satz 1 "`Verantwortliche für Soziales"' geändert zu "`Verantwortliche"';\\
§ 6 Abs. 4 NEU;\\
§ 7 Abs. 1 Datum aktualisiert;\\

Geändert am 08.01.2015 \\
§ 1 Abs. 1 Satz 1 "`und wirtschaftlichen"' ergänzt;\\
§ 2 Abs. 3 Satz 1 "`chronischer Krankheit oder Behinderung"' ergänzt;\\
§ 3 Abs. 1 Satz 1 "`freiwillige Zuwendungen Dritter,"' ergänzt;\\
§ 3 Abs. 3 Satz 1 "`und sonstigen Darlehen"' ergänzt;\\
§ 3 Abs. 3 Satz 2 NEU;\\
§ 3 Abs. 4 Satz 1-4 und 6 NEU;\\
§ 3 Abs. 6 Satz 1 "`eingetragenen"' gestrichen;\\
§ 5 Abs. 2 Satz 1 und 3 NEU;\\
§ 7 Inkrafttreten aktualisiert;

Geändert am 04.06.2015 \\
§ 1 Abs. 1 Satz 1 "`den Semesterbeitrag für das Studentenwerk"' ergänzt;\\


\normalsize
~\\*[4cm]
\begin{center}
\hspace*{\fill}
\parbox{7cm}{Robert Georges\\GF Finanzen}
\hfill\parbox{7cm}{Jan-Malte Jacobsen\\GF Hochschulpolitik}
\hspace*{\fill}
\end{center}