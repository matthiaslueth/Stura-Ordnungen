%\addchap[Richtlinie zur Anerkennung von Hochschulgruppen durch den Studentenrat der TU~Dresden]{Richtlinie zur Anerkennung von Hochschulgruppen\\ durch den Studentenrat der TU Dresden}
\markright{Richtlinie zur Anerkennung von Hochschulgruppen}
\setcounter{section}{0} % ist nötig um den Paragrafenzähler zurücksetzen
\begin{multicols}{2}




\section{Status Hochschulgruppe}

\Abs \Satz Auf Antrag kann eine Gruppe von Studierenden der TU Dresden als Hochschulgruppe im Sinne dieser Richtlinie (im folgenden "`Hochschulgruppe"') anerkannt werden.

\Abs \Satz Über die Anerkennung beschließt der Studentenrat, seine Geschäftsführung oder ein Ausschuss des Studentenrates.

\Abs \Satz Die Anerkennung als Hochschulgruppe wird bis zum Ende der Legislatur ausgesprochen\. Der Antrag muss eine kurze Beschreibung der Gruppe und ihrer Ziele, eine E-Mail-Adresse und nach Möglichkeit Telefonnummer enthalten\. Es müssen Vertreterinnen im Sinne dieser Richtlinie genannt werden\. Die Hochschulgruppe erklärt sich einverstanden, dass ihre E-Mail-Adresse in einen vom Studentenrat moderierten Verteiler aufgenommen wird.

\Abs \Satz Die Anerkennung der Hochschulgruppe kann verweigert werden\. Sie ist insbesondere zu verweigern, wenn 
\begin{enumerate}
\item die Gruppe aus weniger als fünf Mitgliedern besteht,
\item die Gruppe nicht ausschließlich oder zum ganz wesentlichen Teil aus Studierenden zusammengesetzt ist,
\item Zweifel bestehen, dass Studierende die Willensbildung der Gruppe maßgeblich prägen,
\item die Anerkennung der Erfüllung der Aufgaben der Studierendenschaft aus § 74 Abs. 3 SächsHG entgegensteht,
\item die Anerkennung der Erfüllung der Aufgaben der Hochschule aus § 4 SächsHG entgegensteht,
\item die Gruppe entgegen grundsätzlicher Positionen des Studentenrates handelt.
\end{enumerate}
\Satz Sofern Tatsachen später bekannt werden, die der Anerkennung einer Hochschulgruppe entgegenstehen, ist die Anerkennung der Hochschulgruppe gemäß §~49 Abs.~2 Satz~1 VwVfG durch das Plenum des Studentenrates zu widerrufen.

\Abs \Satz Änderungen der Daten sind unverzüglich dem StuRa bekannt zu geben.


\section{Rechte von Hochschulgruppen}

\Abs \Satz Hochschulgruppen können den Materialverleih des Studentenrates nutzen\. Näheres regelt die entsprechende Richtlinie.

\Abs \Satz Hochschulgruppen können auf Wunsch auf der Internetseite des Studentenrates verlinkt werden\. Sie können sich, ihre Projekte und ihre Termine auf der dafür vorgesehenen Internetseite des Studentenrates vorstellen.

\Abs \Satz Hochschulgruppen bekommen die Möglichkeit sich in der Broschüre "`spiritus rector"' des Studentenrates kurz vorzustellen\. Sie können ihre Projekte in der Zeitung des Studentenrates vorstellen\. Sie können sich auf der dafür vorgesehenen Pinnwand im Studentenrat vorstellen.

\Abs \Satz Hochschulgruppen können die Schneidemaschine und den Broschürentacker des Studentenrates nutzen, soweit diese nicht vom Studentenrat selber benötigt werden\. Der Studentenrat kann Flugblätter, Broschüren und Plakate für die Hochschulgruppen verteilen.

\Abs \Satz Die Geschäftsführung des Studentenrates kann Hochschulgruppen bei Anliegen an andere Institutionen unterstützen.

\Abs \Satz Hochschulgruppen können ein Postfach in den Räumlichkeiten des Studentenrates bekommen.



\section{Schlussbestimmungen}
\Abs \Satz Es ergibt sich mit der Anerkennung als Hochschulgruppe kein Rechtsanspruch auf unter §~2 genannte Rechte.



\end{multicols}

\pagebreak
\vspace{1cm}
Inkraftgetreten am 29.~Juni~2006.
\\


\footnotesize
Geändert am 17.~Juli~2008\\
§~1~Abs.~3~S.~4 "`die"' ersetzt durch "`ihre"';\\
§~2~Abs.~1 "`Durchführungsrichtlinie"' ersetzt durch "`Richtlinie"';\\
alt~§~2~Abs.~7 gestrichen;\\
alt~§~2~Abs.~8 "`Punkte"' durch "`Rechte"' ersetzt und als neuer §~3~Abs.~1 aufgeführt;\\
Geändert am 13.~November~2008\\
§~1~Abs.~2 Ausschuss ergänzt;\\
§~1~Abs.~4 NEU;\\
\\
Geändert am 15.~Juli~2010\\
§~1~Abs.~1~S.~4 Korrektur des VwVfG-Verweis und hinzufügen von "`durch das Plenum des Studentenrates"' \\

\normalsize
~\\*[4cm]
\begin{center}
\hspace*{\fill}
\parbox{7cm}{Steven Seiffert\\GF Hochschulpolitik}
\hfill\parbox{7cm}{Matthias Zagermann\\GF Inneres}\hspace*{\fill}
\end{center}