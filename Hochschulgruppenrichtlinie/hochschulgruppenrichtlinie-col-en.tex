%\addchap[Richtlinie zur Anerkennung von Hochschulgruppen durch den Studentenrat der TU~Dresden]{Richtlinie zur Anerkennung von Hochschulgruppen\\ durch den Studentenrat der TU Dresden}
\markright{ Directives on the recognition of student groups} 
\setcounter{section}{0} 

\begin{multicols}{2} 
\section{Status of student group} 
\Abs \Satz Upon application a group of students can be acknowledged as a student group of the TU Dresden in terms of these directives (hereafter referred to as “’student group”’).

\Abs \Satz The student council, its executive board or a committee of the student council decides on the approval. 

\Abs \Satz The approval of a student group lasts until the end of the legislature\. The application must consist of a short description of the group and their goals, an e-mail address and if possible a telephone number\. Representatives in terms of these directives have to be named\. The student group agrees with the acquisition of their e-mail address to a, by the student council generated, mailing list.

\Abs \Satz The approval of a student group can be denied\. It is especially to be denied, if
\begin{enumerate} 
\item the group consists of less than five members,
\item the group is not solely or to the largest part composed of students,
\item doubts exist that the students essentially shape the decision-making process,
\item the approval prevents the fulfillment of the student body’s responsibilities according to § 74 Abs. 3 SächsHG,
\item the approval prevents the fulfillment the university’s responsibilities according to § 4 SächsHG,
\item the group acts in contradiction the fundamental positions of the student council.
\end{enumerate}

\Satz In case facts which exclude an approval of a student group become known later, the approval of the student group has to be revoked, according to §~49 Abs.~2 Satz~1 VwVfG by the plenum of the student council. 

\Abs \Satz Changes of information must be communicated immediately to the student council.

\section{Rights of student groups} 
\Abs \Satz Student groups are able to use the student council’s equipment hire\. The particulars are set out in the appropriate guideline.

\Abs \Satz On request student groups can be linked to the website of the student council\. They are 
able to present their projects and events on a from the student council intended website. 

\Abs \Satz Student groups are given the opportunity to introduce themselves in the brochure ‘spiritus rector’ by the student council\. They are able to present the projects in the student council’s newspaper\. They are able to introduce themselves on the by the student council intended noticeboard.

\Abs \Satz Student groups are able to use the student council’s paper trimmer and long reach stapler, if the student council does not need them itself\. The student council can hand out leaflets, brochures and posters on behalf of the student groups. 

\Abs \Satz The executive board of the student council is able to support the student group’s concerns vis-à-vis other institutions. 

\Abs \Satz The student groups have the right to receive a mailbox in the premises of the student council. 

\section{Final clause} 
48 \Abs \Satz The approval as student group does not result in a legal claim to the rights mentioned in §~2. 

\end{multicols} 

%\pagebreak 

\vspace{1cm} 

Effective from June~29~2006


\normalsize
~\\*[4cm]
\begin{center}
\hspace*{\fill}
\parbox{7cm}{Jan-Malte Jacobsen\\GF Hochschulpolitik}
\hfill\parbox{7cm}{Robert Hoppermann\\GF Finanzen}\hspace*{\fill}
\end{center}