%Header Grundordnung 
%\addchap[Grundordnung]{Grundordnung (GrO)\\der Studentenschaft der TU Dresden}
%\markright{Grundordnung}
%\setcounter{section}{0} % ist nötig um den Paragrafenzähler zurücksetzen


%Header Grundordnung multicols
%\addchap[Grundordnung]{Grundordnung (GrO)\\der Studentenschaft der TU Dresden}
\markright{Grundordnung}
\setcounter{section}{0} % ist nötig um den Paragrafenzähler zurücksetzen
\begin{multicols}{2}

\section*{Vorbemerkung}
\Satz Für den gesamten Text dieser Grundordnung und ihrer Ergänzungsordnungen schließen grammatikalisch feminine Formen zur Bezeichnung von Personen solche weiblichen und männlichen Geschlechts gleichermaßen ein\. Der 															Studentenrat der TU~Dresden wird im folgenden kurz StuRa, sowie die Fachschaftsräte kurz FSR genannt.



\begin{description}
\item[1. Abschnitt] Grundsätze der Studentenschaft
\item[2. Abschnitt] Fachschaften
\item[3. Abschnitt] Studentenrat
\item[4. Abschnitt] Legislative des StuRa
\item[5. Abschnitt] Exekutive des StuRa
\item[6. Abschnitt] Schlussbestimmungen
\end{description}



\section*{1. Grundsätze der Studentenschaft}



\section{Begriffsbestimmung und Rechtsstellung}

\Abs \Satz Alle eingeschriebenen Studentinnen der Technischen Universität Dresden bilden die Studentenschaft\. Jedes gewählte Mitglied der Studentenschaft hat das Recht, die weibliche oder die männliche Bezeichnung ihres Amtes zu führen\. Ausländische 	und staatenlose Studienbewerberinnen, denen befristet bis zum Bestehen bzw. endgültigen Nichtbestehen der Sprachprüfung oder 		der Feststellungsprüfung die Rechtsstellung von Studentinnen der TU~Dresden verliehen worden ist, werden im Rahmen dieser 			Grundordnung wie eingeschriebene Studentinnen behandelt.
  
\Abs \Satz Die Studentenschaft ist eine rechtsfähige Teilkörperschaft der Universität.
  
\Abs \Satz Sie ordnet im Rahmen der gesetzlichen Regelungen, der Grundordnung der Universität und dieser Grundordnung ihre Angelegenheiten selbstständig.
  
\Abs \Satz Sie hat das Recht, sich mit Studentenschaften anderer Hochschulen zu einem Verband zusammenzuschließen.



\section{Aufgaben der Studentenschaft}

\Abs \Satz Die Studentenschaft hat folgende Aufgaben:
\begin{enumerate}
\item Vertretung der Interessen ihrer Mitglieder als Angehörige der Universität,
\item Wahrnehmung der wirtschaftlichen und sozialen Belange einschließlich der sozialen Selbsthilfe ihrer Mitglieder und Stellungnahme zu diesbezüglichen Fragen,
\item Wahrnehmung der fachlichen Belange ihrer Mitglieder und Stellungnahme zu diesbezüglichen Fragen,
\item Unterstützung der kulturellen und sportlichen Interessen ihrer Mitglieder,
\item Pflege der überörtlichen und internationalen Studentinnenbeziehungen,
\item Förderung der politischen Bildung und des staatsbürgerlichen Verantwortungsbewusstsein der Studentinnen, fern jeglicher parteipolitischer Bindung.
\end{enumerate}

\Abs \Satz Zur Durchführung ihrer Aufgaben erhebt die Studentenschaft von ihren Mitgliedern Beiträge.



\section{Rechte und Pflichten der Mitglieder}
\Abs \Satz Jede Studentin hat das Recht, an der Studentischen Selbstverwaltung mitzuwirken.

\Abs \Satz Alle Mitglieder der Studentenschaft sind berechtigt, Anfragen an die Organe der Studentenschaft gemäß §~5 zu stellen\. Ferner hat jedes Mitglied das Recht Anträge an die beschlussfassenden Organe nach §~5 zu stellen.

\Abs \Satz Jedes Mitglied der Studentenschaft hat die Pflicht zur Beitragszahlung nach Maßgabe der jeweils gültigen Beitragsordnung.

\Abs \Satz Diese Grundordnung sowie alle ihre Ergänzungsordnungen sind für die Mitglieder der Studentenschaft verbindlich.



\section{Studentenbefragung}

\Abs \Satz Der StuRa kann in Angelegenheiten nach §~16,~Abs.~2,~Nr.~1~bis~3 mit \nicefrac{2}{3}~Mehrheit der Mitglieder eine Befragung der Studentenschaft beschließen.

\Abs \Satz Eine Befragung findet ebenfalls statt, wenn es in schriftlicher Form von fünf Prozent der Mitglieder der Studentenschaft beantragt wird\. Die Organisation der Befragung obliegt in diesem Fall den Antragstellerinnen\. Die Kosten trägt grundsätzlich der StuRa.

\Abs \Satz Die Befragung wird innerhalb von vier Vorlesungswochen nach Beschlussfassung des StuRa bzw. nach Antragstellung gemäß Abs. 2 an fünf aufeinander folgenden Vorlesungstagen von einer zu bildenden Kommission, in die der StuRa Vertreterinnen entsenden kann, durchgeführt.

\Abs \Satz Die Befragung erfolgt unmittelbar, allgemein, frei, gleich und geheim.

\Abs \Satz Das Ergebnis der Befragung dient dem StuRa bei seinem weiterem Handeln als Leitlinie, wenn sich mindestens 30~\% der Mitglieder der Studentenschaft an der Befragung beteiligten.


%Einfügen Paragraph 4a der nicht in die Nummerierung passt. Sollte unbedingt irgendwann korrigiert werden.
\setcounter{section}{3}
\section{a Anfragen}
\Abs \Satz Anfragen an die Organe der Studentenschaft sind von diesen binnen 14 Tagen zu beantworten\. Dies hat auf Wunsch schriftlich zu erfolgen\. Ist eine fristgerechte Beantwortung nicht möglich, so ist die der Anfragenden eine Begründung über den Grund der Verzögerung abzugeben.
\setcounter{section}{4}



\section{Die Organe}

\Abs \Satz Beschlussfassende Organe der Studentenschaft sind:
\begin{enumerate}
\item der Studentenrat,
\item die Geschäftsführung,
\item der Sitzungsvorstand und
\item ggf. die Ausschüsse.
\end{enumerate}

\Abs \Satz Die beschlussfassenden Organe der Fachschaft sind:
\begin{enumerate}
\item der Fachschaftsrat,
\item die Vertreterinnen der Fachschaft im Studentenrat und
\item ggf. die Fachschaftsvollversammlung.
\end{enumerate}

\Abs \Satz Neben diesen Organen werden als Grundordnungsorgane mit beratender Kompetenz eingerichtet:
\begin{enumerate}
\item die Referate und
\item die Arbeitsgemeinschaften.
\end{enumerate}

\Abs \Satz Im Rahmen der AE-Ordnung werden Ämter der Exekutive wie folgt definiert:
\begin{enumerate}
\item Referatsmitarbeiterinnen handeln im Auftrag der jeweiligen Referentinnen o der Geschäftsführerinnen,
\item Referentinnen stehen einem Referat vor, haben einen klar abgegrenzten Aufgabenbereich, handeln nach Tätigkeitsbeschreibung,
\item Geschäftsführerinnen leiten ihren Geschäftsbereich an, vertreten den StuRa nach außen und fällen Beschlüsse zwischen den StuRa Sitzungen.
\end{enumerate}

%Einfügen Paragraph 5a der nicht in die Nummerierung passt. Sollte unbedingt irgendwann korrigiert werden.
\setcounter{section}{4}
\section{a Beschlussfähigkeit}
\Abs \Satz Die Beschluss fassenden Organe der Studentenschaft nach § 5 (1) sind beschlussfähig, wenn die Sitzung ordnungsgemäß einberufen wurde und mehr als die Hälfte der Mitglieder mit aktivem Stimmrecht anwesend sind\.
\setcounter{section}{5}

\section*{2. Fachschaften}



\section{Gliederung}

\Abs \Satz Die Studentenschaft gliedert sich in die folgenden Fachschaften:
\begin{enumerate}
\item Mathematik
\item Physik
\item Psychologie
\item Chemie/Lebensmittelchemie
\item Biologie
\item der Philosophischen Fakultät
\item Sprach-, Literatur- und Kulturwissenschaften
\item Allgemeinbildende Schulen
\item Sozialpädagogik/Erziehungswissenschaften (M. A.)
\item Berufspädagogik
\item Jura
\item Wirtschaftswissenschaften
\item Informatik
\item Elektrotechnik
\item Maschinenwesen
\item Bauingenieurwesen
\item Architektur/Landschaftsarchitektur
\item Forstwissenschaften
\item Geowissenschaften
\item Hydrowissenschaften
\item Verkehrswissenschaften "'Studentenschaft Friedrich List"'
\item Medizin
\item IHI Zittau '"Studierendenschaft IHI"'
\end{enumerate}



\section{Grundsätzliches}

\Abs \Satz Die Fachschaft ist eine rechtsfähige Teilkörperschaft der TU Dresden und ihrer Studentenschaft.

\Abs \Satz Sie ordnet im Rahmen der gesetzlichen Regelungen, der Grundordnung der TU Dresden und der Grundordnung der Studentenschaft ihre Angelegenheiten selbst\. Neben den Aufgaben nach §~2 fördert die Fachschaft die fachlichen Interessen der Studentinnen und betreut deren Studienangelegenheiten.

\Abs \Satz Gehören einer Fakultät mehrere Fachschaften an, bilden diese einen Konvent\. Soweit nicht anders geregelt, entsenden die FSR dafür jeweils drei Delegierte.

\Abs \Satz Jedes Mitglied der Studentenschaft ist Mitglied in genau einer Fachschaft.



\section{Zusammensetzung des Fachschaftsrat}

\Abs \Satz Der Fachschaftsrat wird von den Mitgliedern der Fachschaft nach Maßgabe der Wahlordnung der TU Dresden auf ein Jahr gewählt\. Die Mitgliedschaft im FSR endet durch Rücktritt, Exmatrikulation oder Tod.

\Abs \Satz Die Anzahl der zu wählenden Mitglieder eines FSR wird durch Beschluss des FSR festgelegt\. Sie beträgt mindestens drei, jedoch höchstens fünfundzwanzig.

\Abs \Satz Wird in einer Fachschaft kein FSR gewählt, kann der StuRa diese Fachschaft vertreten.



\section{Aufgaben und Funktionen des FSR}

\Abs \Satz Der FSR vertritt die Studentinnen einer Fachschaft im Rahmen seiner Aufgaben nach §~7~Abs.~2.

\Abs \Satz Der FSR entsendet seine Vertreterinnen in den Studentenrat.

\Abs \Satz Rechtsgeschäftliche Erklärungen müssen von mindestens zwei Mitgliedern des Fachschaftsrates gemeinschaftlich abgegeben werden.



\section{Fachschaftsordnung}

\Abs \Satz Der FSR kann sich im Rahmen des SächsHG, der Wahlordnung der TU Dresden und der Grundordnung der Studentenschaft eine Fachschaftsordnung geben.

\Abs \Satz Die Fachschaftsordnung trifft insbesondere Regelungen über Zusammensetzung, Organe und Beschlussfassung des FSR.

\Abs \Satz Beschluss und Änderung der Fachschaftsordnung bedürfen einer \nicefrac{2}{3}~Mehrheit der Mitglieder des Fachschaftsrates.

\Abs \Satz Die Fachschaftsordnung kann eine Fachschaftsvollversammlung vorsehen.

\Abs \Satz Fachschaftsordnungen und deren Änderungen treten nach Kenntnisnahme durch die Geschäftsführung des StuRa in Kraft, wenn diese keine berechtigten Zweifel an der Rechtmäßigkeit vorbringt.

\Abs \Satz In Fachschaften ohne Fachschaftsordnung oder für nicht geregelte Angelegenheiten gilt die Geschäftsordnung des StuRa entsprechend.



\section{Finanzen}

\Abs \Satz Die Fachschaften verwalten die ihnen übertragenen und selbst erwirtschafteten Mittel selbständig nach Maßgabe der Finanzordnung der Studentenschaft und verwenden sie ausschließlich für ihre Grundordnungsgemäßen Aufgaben.

\Abs \Satz Der FSR ist dem StuRa über die Verwendung seiner Gelder rechenschaftspflichtig.


\section*{3. Studentenrat}



\section{Legislatur und Amtsperioden}

\Abs \Satz Die Legislatur des StuRa beginnt mit seiner Konstituierung.

\Abs \Satz Die Amtsperiode aller Wahlämter des StuRa dauert ein Jahr, von Beginn des Sommersemester bis Ende des darauf folgenden Wintersemesters\. Ausnahme hiervon sind die Vertreterinnen des StuRa im Verwaltungsrat des Studentenwerkes.

\Abs \Satz Als Amtsträgerinnen gelten die vom StuRa gewählten Personen\. Jede Amtsträgerin kann zurücktreten\. Der Rücktritt muss schriftlich erfolgen und auf einer Sitzung des StuRa bekannt gemacht werden, gleiches gilt für Mitglieder von Referaten.

\Abs \Satz Die Abwahl einer Amtsträgerin ist nur durch ein Misstrauensvotum der Mehrheit der Mitglieder des StuRa möglich.

\Abs \Satz Amtsträgerinnen müssen voll geschäftsfähig im Sinne des Bürgerlichen Gesetzbuches (BGB) sein.

\Abs \Satz Jede Amtsträgerin hat einen Anspruch auf Weiterbildung sofern sich diese auf deren Aufgabenbereich bezieht.

\Abs \Satz Amtsträgerinnen können nur an der TU Dresden immatrikulierte Studentinnen sein.



\section{Rechtsgeschäftliche Erklärungen}

\Abs \Satz Rechtsgeschäftliche Erklärungen bedürfen eines StuRa-Beschlusses und der Schriftform\. Sie sind von zwei Geschäftsführerinnen zu unterzeichnen.

\Abs \Satz Entsprechen rechtsgeschäftliche Erklärungen dem Aufgabenbereich einer Referentin die zugleich Mitglied des StuRa ist, kann diese anstelle der zweiten Geschäftsführerin unterzeichnen.



\section{Angestellte}

\Abs \Satz Der StuRa beschäftigt eine Angestellte als Kassenwärtin.

\Abs \Satz Einrichtung und Abschaffung von Stellen zur hauptberuflichen Beschäftigung müssen vom StuRa beschlossen werden.

\Abs \Satz Über Einstellung und Entlassung von hauptberuflich Beschäftigten entscheidet der StuRa\. Die Bedingungen des Beschäftigungsverhältnisses richten sich nach TV-L~(Tarifgebiet~Ost).

\Abs \Satz Die Angestellten haben das Recht, aus der Mitte des Studentenrates eine Vertrauensperson für die laufende Legislatur zu bestimmen, die Ansprechpartnerin für Probleme mit der Dienstvorgesetzten ist.


\section*{4. Legislative des StuRa}



\section{Zusammensetzung des StuRa}

\Abs \Satz Der StuRa setzt sich aus den von den einzelnen FSR entsandten Vertreterinnen zusammen\.

\Abs \Satz Der StuRa hat maximal 39~Sitze, die wie folgt besetzt werden:
\begin{enumerate}
\item Jeder FSR entsendet eine Vertreterin (Basisvertreterin).
\item Entsprechend der Größe der jeweiligen Fachschaft können zusätzlich bis zu drei Vertreterinnen (weitere Vertreterinnen) nach folgendem Verfahren entsandt werden. Es werden pro Fachschaft drei Kennzahlen durch Multiplikation der Anzahl der Fachschaftsmitglieder mit 30,~17,~7 und anschließender Division durch die Anzahl der Mitglieder der Studentenschaft gebildet. Anhand der Kennzahlen größer Eins werden nach dem Höchstzahlverfahren die weiteren Vertreterinnen bis zur maximalen Größe des Studentenrates von 33~Basis- und weiteren Vertreterinnen entsandt.
\item Geschäftsführerinnen werden zu Vertreterinnen mit besonderem Sitz (besondere Vertreterinnen), wenn der FSR die maximal mögliche Zahl an Basis- und weiteren Vertreterinnen entsandt hat. Ist die Geschäftsführerin Basis- oder weitere Vertreterin, kann der FSR eine Vertreterin neu entsenden.
\item Eine Fachschaft darf insgesamt nicht mehr als fünf Vertreterinnen haben.
\end{enumerate}

\Abs \Satz Entsendet ein FSR weniger weitere Vertreterinnen als ihm das nach Abs.~2~Nr.~2 möglich ist, geht die Möglichkeit der Entsendung dieser Vertreterinnen nach zwei aufeinanderfolgenden Sitzungen an die nach dem Höchstzahlverfahren gemäß Abs.~2~Nr.~2 nachfolgenden Fachschaften über.

\Abs \Satz Nimmt eine Vertreterin an zwei aufeinanderfolgenden Sitzungen unentschuldigt nicht teil, ruht ihr Mandat für die Zeit ihrer weiteren Abwesenheit\. Ruhende Mandate weiterer Vertreterinnen werden wie Nichtentsendungen nach Abs.~3 behandelt\. Mitglieder, deren Mandat ruht, besitzen kein aktives Stimmrecht.

\Abs \Satz Nach Rücktritt oder Abwahl einer Geschäftsführerin hat der entsprechende FSR alle Vertreterinnen neu zu entsenden.

\Abs \Satz Fachschaftsräte, die in der ablaufenden Amtsperiode mindestens eine Geschäftsführerin gestellt haben und/oder in der folgenden Amtperiode mindestens eine Geschäftsführerin stellen, müssen zur ersten Sitzung des Sommersemesters eine neue Entsendung vornehmen.

\Abs \Satz Die Mitgliedschaft einer Vertreterin im StuRa endet mit dem Ende der Legislatur des StuRa\. Fernen endet sie durch Rücktritt, Exmatrikulation, Tod oder Rücknahme der Entsendung durch den FSR.


%Einfügen Paragraph 15a der nicht in die Nummerierung passt. Sollte unbedingt irgendwann korrigiert werden.
\setcounter{section}{14}
\section{a Beratende Mitglieder}
\Abs \Satz Die Referentin Ausländische Studierende ist qua Amt Beratendes Mitglied des Studentenrats.
\setcounter{section}{15}


\section{Aufgaben und Funktionen des StuRa} % §16

\Abs \Satz Der StuRa ist das oberste beschlussfassende Organ der Studentenschaft\. Es bringt den Willen der Studentenschaft zum Ausdruck.

\Abs \Satz Der StuRa hat folgende Aufgaben:
\begin{enumerate}
\item Richtlinien für die Erfüllung der Aufgaben der Studentenschaft zu beschließen,
\item in fakultätsübergreifenden Angelegenheiten der Studentenschaft zu beschließen,
\item die Amtsträgerinnen des StuRa zu wählen und von ihnen Rechenschaft entgegenzunehmen,
\item die Entsendung von Mitgliedern in die Referate,
\item die Vertreterinnen der Studentenschaft in sonstige, die Gesamtinteressen der Studentenschaft berührende Einrichtungen und Organe zu entsenden bzw. zu nominieren, sofern dem nicht andere Bestimmungen entgegenstehen,
\item das Arbeitsprogramm und den Haushalt beschließen,
\item die Grundordnung der Studentenschaft und deren Ergänzungsordnungen zu beschließen,
\end{enumerate}

\Abs \Satz Die Mitglieder des StuRa haben das Recht zur Einsicht in Unterlagen der Geschäftsführung.

\Abs \Satz Die Mitglieder des StuRa sind verpflichtet, ihre Aufgaben ehrenamtlich nach bestem Wissen und Gewissen zu erfüllen.



\section{Öffentlichkeit}

\Abs \Satz Der StuRa verhandelt in öffentlichen Sitzungen.

\Abs \Satz Jedes Mitglied der Studentenschaft hat Rede- und Antragsrecht.

\Abs \Satz Die Protokolle der StuRa-Sitzungen sind zu veröffentlichen.

\Abs \Satz Ausnahmen hiervon bestehen nur im Rahmen der Geschäftsordnung.



\section{Stimmrechte}

\Abs \Satz Jedes StuRa-Mitglied kann jeweils nur eine Stimme wahrnehmen\. Eine Vertretung ist nicht statthaft.

\Abs \Satz Ausnahme von Abs. 1 ist die Fachschaft Forstwissenschaften\. Sie kann eine Stellvertreterin ihrer entsandten Vertreterin ernennen\. Dieser Absatz tritt außer Kraft, wenn die Fachschaft Forstwissenschaften mehr als eine Vertreterin entsenden darf oder ihr Sitz nicht mehr in Tharandt ist.

\Abs \Satz Ausnahme von Abs. 1 ist die Fachschaft IHI Zittau\. Sie kann eine Stellvertreterin ihrer entsandten Vertreterin ernennen\. Dieser Absatz tritt außer Kraft, wenn die Fachschaft IHI Zittau mehr als eine Vertreterin entsenden darf oder ihr Sitz nicht mehr in Zittau ist.


\section{Mehrheiten}

\Abs \Satz Im Rahmen dieser Grundordnung und ihrer Ergänzungsordnungen gelten folgende Mehrheiten:
\begin{enumerate}
\item Einfache Mehrheit (Mehrheit der anwesenden Mitglieder);
\item Mehrheit der Mitglieder (Mehrheit der aktiven Stimmrechte);
\item \nicefrac{2}{3}~Mehrheit der Mitglieder (\nicefrac{2}{3}~der aktiven Stimmrechte).
\end{enumerate}

\Abs \Satz Im Rahmen der Geschäftsordnung gilt anstatt der Mehrheit der Mitglieder die \nicefrac{2}{3}~Mehrheit der anwesenden Mitglieder.

\Abs \Satz Der StuRa entscheidet grundsätzlich mit einfacher Mehrheit sofern Grundordnung und Ergänzungsordnungen keine andere Mehrheit vorschreiben.



\section{Beschlussfähigkeit und Beschlussfassung}

\Abs \Satz Der StuRa ist beschlussfähig, wenn mehr als die Hälfte seiner Mitglieder mit aktivem Stimmrecht anwesend ist.

\Abs \Satz Beschlüsse des StuRa werden, wenn von diesem nichts anderes bestimmt wird, mit der Beschlussfassung wirksam.

\Abs \Satz Der StuRa kann in seiner Amtsperiode gefasste Beschlüsse nur mit einer höheren Mehrheit gemäß §~19~Abs.~1 ändern oder aufheben; bei früheren Beschlüssen mit Ausnahme von \S~29~Abs.~3 genügt eine einfache Mehrheit.

\Abs \Satz Beschlüsse, die den Studentenrat finanziell über das Haushaltsjahr hinaus binden, sowie Grundordnungsänderungen bedürfen eines Beschlusses auf einer ordentlichen Sitzung.

\Abs \Satz Beschlüsse eines beschlussfassenden Organs der Studentenschaft mit Ausnahme des StuRa werden wirksam, wenn auf der folgenden, ordentlichen, beschlussfähigen Sitzung des StuRa das Protokoll vorliegt und diesen nicht durch einen Antrag auf Neubefassung nach §10 (6) Geschäftsordnung widersprochen wird.



\section{Ordentliche Sitzungen}

\Abs \Satz Ordentliche Sitzungen des StuRa finden in der nicht vorlesungsfreien Zeit alle zwei Wochen gemäß der Geschäftsordnung statt.

\Abs \Satz In der vorlesungsfreien Zeit finden maximal drei ordentliche Sitzungen statt, zwischen denen jeweils maximal vier Wochen liegen.

\Abs \Satz Kann eine Sitzung aufgrund eines Feiertages oder eines sonstigen vorlesungsfreien Tages nicht regulär stattfinden, wird sie um eine Woche vorgezogen\. Alle nachfolgenden Sitzungstermine verschieben sich entsprechend.

\Abs \Satz Im Juni eines Jahres werden die Termine für die ordentlichen Sitzungen der folgende Amtsperiode des StuRa veröffentlicht\. Dabei sind die Termine für die Rechenschaftsberichte festzulegen.



\section{Außerordentliche Sitzungen}

\Abs \Satz Zusätzlich zu den ordentlichen StuRa-Sitzungen sind auf Beschluss des StuRa, des Sitzungsvorstands, der Geschäftsführung oder auf Initiative von mindestens \nicefrac{1}{3} der Mitglieder des StuRa Sondersitzungen möglich.

\Abs \Satz Auf außerordentlichen Sitzungen darf nur zu den auf der Einladung enthaltenen Themen diskutiert und beschlossen werden.

\Abs \Satz In der vorlesungsfreien Zeit beträgt die Ladungsfrist für außerordentlichen Sitzungen 10~Tage\. Sie reduziert sich in der nicht vorlesungsfreien Zeit auf 72~Stunden.



\section{Der Sitzungsvorstand}

\Abs \Satz Der Sitzungsvorstand besteht aus drei vom StuRa gewählten Mitgliedern\. Zusätzlich ist die Referentin Struktur Mitglied des Sitzungsvorstandes.

\Abs \Satz Der Sitzungsvorstand leitet und strukturiert die Sitzung des StuRa\. Er ist dafür verantwortlich, dass sämtliche Unterlagen für die Sitzung rechtzeitig bereitstehen\. Näheres regelt die Geschäftsordnung.

\Abs \Satz Der Sitzungsvorstand bestimmt die Versammlungsleiterin in der Regel aus seiner Mitte\. Die Versammlungsleiterin hat die Ordnungsgewalt auf der Sitzung des StuRa\. Ihr obliegt die Auslegung der Grundordnung und Ordnungen mit Wirkung für den Verlauf der aktuellen Sitzung\. Auf außerordentlichen Sitzungen hat die Versammlungsleiterin insbesondere das Recht, Initiativen abzulehnen, die §~22~Abs.~2~und~§~20~Abs.~3 zuwiderlaufen.

\Abs \Satz Der Sitzungsvorstand ist für die Erstellung, Veröffentlichung und Verwaltung des Protokolls zuständig.

\Abs \Satz Ruht das Mandat eines Mitgliedes des StuRa gemäß §~15~Abs.~4~S.~1, hat der Sitzungsvorstand unverzüglich den entsprechenden FSR zu informieren.


%Einfügen Paragraph 23a der nicht in die Nummerierung passt. Sollte unbedingt irgendwann korrigiert werden.
\setcounter{section}{22}
\section{a Referentin Struktur}

\Abs \Satz Die Referentin Struktur ist qua Amt Mitglied im Sitzungsvorstand.

\Abs \Satz Sie ist zuständig für:
\begin{enumerate}
\item Die Berechnung der Sitze der Fachschaften im StuRa nach Grundordnung,
\item Überprüfung der Entsendungen in den Studentenrat,
\item die Information der FSR über ruhende Mandate gemäß § 15, Abs. 4, Satz 1,
\item die Überwachung der Begründungen und Entscheidungen des StuRa auf Konformität mit Ordnungen der Studentenschaft,
\item die Überwachung der Ordnungen der Studentenschaft auf Änderungsbedarf,
\item die Archivierung der Protokolle sowie der Grundordnung und der weiteren Ordnungen des StuRa,
\item Erfassung und Verwaltung der Kontaktdaten der StuRa-Mitglieder und Mitarbeiter/innen,
\item die Verwaltung der Mailinglisten, E-Mail-Verteiler und Weiterleitungen sowie
\item die Ausschreibung der Posten und Aktualisierung der Struktur und Tätigkeitsbeschreibungen.
\end{enumerate}


\setcounter{section}{23}


\section{Ausschüsse} % §24

\Abs \Satz Ein Ausschuss besteht aus 4 bis 7 Mitgliedern des StuRa, welche zum Zeitpunkt ihrer Wahl über das aktive Stimmrecht im StuRa verfügen\. Sie werden vom Studentenrat für die laufende Legislatur der Legislative gewählt\.

\Abs \Satz Ausschüsse können mit der Mehrheit der Mitglieder zu Teilaufgaben des StuRa, die dieser mit einfacher Mehrheit beschließen kann, eingerichtet werden\. Dabei müssen Name, Laufzeit, Aufgaben, Sitzungsturnus und gegebenenfalls Sonderregelungen zur Besetzung festgelegt werden.

\Abs \Satz Die Abschaffung eines Ausschusses erfolgt mit der Mehrheit der Mitglieder ungeachtet § 20 Abs. 3 \. Dies gilt nicht für in der Grundordnung festgeschriebene Ausschüsse.

\Abs \Satz Es kann ständige und nichtständige Ausschüsse geben\. Ein ständiger Ausschuss ist ein vom StuRa unbefristet eingerichteter Ausschuss, ein nichtständiger Ausschuss wird für eine bestimmte Zeit eingerichtet.

\Abs \Satz Die Sitzungen sind zu protokollieren, dabei ist § 18, Abs. 3 GO einzuhalten\. Das Protokoll ist den StuRa-Mitgliedern zugänglich zu machen\. Es gelten die Fristen nach § 5 GO. Die Protokolle sind zu veröffentlichen.



%Einfügen Paragraph 24a und b der nicht in die Nummerierung passt. Sollte unbedingt irgendwann korrigiert werden.
\setcounter{section}{23}
\section{a Förderausschuss}

\Abs \Satz Der Förderausschuss ist ein ständiger Ausschuss\. Er tagt in der Vorlesungszeit wöchentlich, in der vorlesungsfreien Zeit in einem regelmäßigen, zuvor zu veröffentlichendem Rhythmus.

\Abs \Satz Der Förderausschuss setzt sich aus der Geschäftsführerin Finanzen, sowie vier bis sechs weiteren, gemäß §24 Abs.1 gewählten Mitgliedern zusammen.

\Abs \Satz Die Aufgaben des Förderausschusses ergeben sich aus der Richtlinie über die finanzielle Förderung studentischer Projekte.

\Abs \Satz Das Protokoll enthält zusätzlich zu den Bestimmungen nach § 18, Abs. 3 Geschäftsordnung die Finanzaufstellungen der Antragsteller.

\Abs \Satz Mitglieder des Förderausschusses dürfen monatlich gemäß den Bestimmungen der AE- Ordnung Aufwandsentschädigung in Höhe von bis zu 20 Euro beantragen. 

\Abs \Satz Sind Mitglieder des Förderausschusses auch in einem anderem Sinne gemäß der AE- Ordnung AE- berechtigt, bleiben die in der AE- Ordnung geltenden Bestimmungen von Abs. 5 unberührt.


\setcounter{section}{24}

\section*{5. Exekutive des StuRa}



\section{Referate} % §25

\Abs \Satz Ein Referat setzt sich aus einer oder mehreren Referentinnen sowie ihren Mitarbeiterinnen zusammen\. Referate werden durch Beschluss vom StuRa zu abgrenzbaren Aufgabenbereichen eingerichtet.

\Abs \Satz Die Referentinnen werden vom StuRa gewählt, die Referats-Mitglieder vom StuRa entsendet.

\Abs \Satz Die Referentin leitet ihr Referat an und trägt die Verantwortung für die Arbeit des Referats\. Sie ist die Ansprechpartnerin des Referats.

\Abs \Satz Die Referate setzen das Arbeitsprogramm und die Beschlüsse des StuRa um.

\Abs \Satz Die Referentinnen sollen auf den Sitzungen der Geschäftsführung anwesend sein.



\section{Geschäftsbereiche}

\Abs \Satz Ein Geschäftsbereich setzt sich aus einer Geschäftsführerin und ein oder mehreren Referaten zusammen\. Jedes Referat wird einem Geschäftsbereich zugeordnet\. Geschäftsbereiche werden durch Beschluss des StuRa eingerichtet.

\Abs \Satz Geschäftsführerinnen werden vom StuRa gewählt\. Sie müssen für die Dauer ihrer Amtsperiode in den StuRa entsendet sein, gegebenenfalls unberührt von §~15~Abs.~2~Nr.~2 auch zusätzlich.

\Abs \Satz Die Geschäftsführerin leitet ihren Geschäftsbereich an und trägt die Verantwortung für die Arbeit und die Erstellung des vierteljährlichen Rechenschaftsberichtes\. Sie ist die Ansprechpartnerin des Geschäftsbereichs.



\section{Geschäftsführung}

\Abs \Satz Die Geschäftsführung setzt sich aus mindestens drei Geschäftsführerinnen zusammen\. Sie kann innerhalb ihrer Aufgaben Beschlüsse fassen.

\Abs \Satz Sie führt die laufenden Geschäfte des StuRa und koordiniert die Arbeit der Geschäftsbereiche und Referate.

\Abs \Satz Die Geschäftsführung vertritt den StuRa und setzt seine Beschlüsse um\. Zwischen den Sitzungen des StuRa fasst Sie nicht aufschiebbare Beschlüsse.

\Abs \Satz Aus ihrer Mitte bestimmt die Geschäftsführung eine Dienstvorgesetzte der Angestellten.

\Abs \Satz Die Geschäftsführung ist dem StuRa zur Rechenschaft verpflichtet.



%Einfügen Paragraph 27a der nicht in die Nummerierung passt. Sollte unbedingt irgendwann korrigiert werden.
\setcounter{section}{26}
\section{a Dienstvorgesetze}

\Abs \Satz Dienstvorgesetzte der Angestellten ist eine Geschäftsführerin.

\Abs \Satz Die Dienstvorgesetzte ist unter anderem zuständig für:
\begin{enumerate}
\item Lohnanweisung,
\item Urlaubsgenehmigung,
\item Festlegung der Arbeitszeit,
\item Weiterbildungsmaßnahmen,
\item Dienstbesprechungen,
\item Arbeitsschutz,
\item Anpassung des Tätigkeitsprofils und des Arbeitsvertrages sowie
\item Erstellung und Aushändigung von schriftlichen Dienstanweisungen.
\end{enumerate}

\Abs \Satz Dienstbesprechungen zwischen den Angestellten und der Dienstvorgesetzten finden monatlich statt\. Diese sind zu protokollieren und in der Personalakte abzulegen.

\Abs \Satz Dienstanweisungen sind von der Geschäftsführung zu beschließen\. Die Dienstvorgesetzte händigt diese schriftlich den Angestellten aus und legt eine Kopie in der Personalakte ab.

\setcounter{section}{27}



\section{Arbeitsgemeinschaften} %28

\Abs \Satz Eine Arbeitsgemeinschaft (AG) ist ein durch den StuRa bestätigter und unterstützter Zusammenschluss von Mitgliedern der Studentenschaft, der innerhalb der Aufgaben gemäß §~74~Abs.~3 SächsHG arbeitet.

\Abs \Satz Eine AG ist inhaltlich nicht an Beschlüsse des StuRa gebunden.

\Abs \Satz Die Arbeitsgemeinschaft kann sich jederzeit selbst auflösen.

\Abs \Satz Der StuRa kann durch Beschluss den Status der Zugehörigkeit der Arbeitsgemeinschaft zum Studentenrat aufheben.

\Abs \Satz Die AG wählt aus ihrer Mitte eine Leiterin und zeigt sie dem StuRa an\. Die AG kann ihre Angelegenheiten durch eine Grundordnung regeln, welche nach Bestätigung durch den StuRa in Kraft tritt.

\Abs \Satz Innerhalb ihres Arbeitsbereiches darf sie sich als "`AG des Studentenrates"' selbstständig in der Öffentlichkeit äußern\. Dabei vertritt sie die Meinung der Mitglieder der AG.

\Abs \Satz Eine AG hat als solche Rede- und Antragsrecht auf einer StuRa-Sitzung.

\Abs \Satz Einer AG kann entgegen §~2~Abs.~1~Nr.~1 dieser Grundordnung gestattet werden, ihren Arbeitsbereich auch auf andere Hochschulen auszudehnen, wenn die Studentenschaft der entsprechenden Hochschule zustimmt.

\Abs \Satz Einzelne Mitglieder der AG können bevollmächtigt werden, eine Geschäftsführerin bei rechtsgeschäftlichen Erklärungen gemäß §~13~Abs.~1 zu vertreten\. Die Vollmacht ist inhaltlich und finanziell zu begrenzen.

\setcounter{section}{27}
\section{b Projekte des Studentenrates}


\Abs \Satz Ein Projekt des Studentenrates (StuRa-Projekt) ist ein vom Studentenratsplenum beschlossenes einmaliges Vorhaben. Ein StuRa Projekt übernimmt außerordentliche Aufgaben, die von der Struktur des StuRa nicht oder nur unzureichend abgebildet werden können\.

\Abs \Satz Bei der Einrichtung ist das Ziel des Projekts zu benennen\.

\Abs \Satz Ein StuRa-Projekt ist befristet, kann aber verlängert werden. Bei absehbarer Langfristigkeit soll die Integration der Aufgaben in die Struktur des StuRa geprüft werden\.

\Abs \Satz Ein StuRa-Projekt ist einer Geschäftsführerin zugeordnet\.

\Abs \Satz Es ist eine Projektsprecherin zu benennen, welche das Projekt gegenüber dem StuRa vertritt und Ansprechpartnerin ist. Weitere Projektmitarbeiterinnen sind ebenfalls zu benennen\.

\Abs \Satz Die Zahl der Mitarbeiterinnen eines StuRa-Projekts kann begrenzt werden\.

\Abs \Satz Insbesondere zum Abschluss des Projektes muss dem StuRa über die Arbeit der Projektgruppe berichtet werden. In dem Bericht sind ebenfalls die aufgewandten Mittel aufzuführen\.

\Abs \Satz Änderungen an Beschlüssen zu StuRa-Projekten werden abweichend von § 20, Absatz 3 stets mit einfacher Mehrheit beschlossen, wenn sie ausschließlich Antragsbestandteile nach den Punkten (3), (5) oder (6) betreffen\.


\setcounter{section}{28}


\section*{6. Schlussbestimmungen}



\section{Ergänzungsordnungen und Richtlinien}

\Abs \Satz Zur Ergänzung dieser Grundordnung beschließt der StuRa mit \nicefrac{2}{3}~Mehrheit seiner gewählten Mitglieder folgende Ergänzungsordnungen:
\begin{enumerate}
\item Finanzordnung der Studentenschaft
\item Beitragsordnung der Studentenschaft
\item Geschäftsordnung des StuRa
\item Härtefallordnung
\item Die AE-Ordnung der Studentenschaft
\item Die Mitgliedschaftsordnung der Studentenschaft
\end{enumerate}

\Abs \Satz Diese sind Bestandteile dieser Grundordnung.

\Abs \Satz Darüber hinaus kann der StuRa mit einfacher Mehrheit Beschlüsse zu Richtlinien und Durchführungsbestimmungen fassen.



\section{Grundordnungsänderung}

\Abs \Satz Als Grundordnungsänderung ist jede Änderung dieser Grundordnung und ihrer Ergänzungsordnungen anzusehen. Grundordnungsänderungen können vom StuRa nur mit \nicefrac{2}{3}~Mehrheit seiner Mitglieder beschlossen werden.



\section{Teilnichtigkeit}

\Abs \Satz Bei Nichtigkeit einzelner Bestimmungen dieser Grundordnung oder ihrer Ergänzungsordnungen gelten die übrigen Bestimmungen fort.



\section{Veröffentlichung}

\Abs \Satz Die Grundordnung der Studentenschaft und ihre Ergänzungsordnungen sowie Änderungen sind öffentlich innerhalb der Studentenschaft bekannt zu machen und jederzeit einsehbar.



\section{Übergangsbestimmungen}

\Abs \Satz Die zum Zeitpunkt der Eingliederung der Fachschaftsrahmenordnung in die Grundordnung gültigen Fachschaftsordnungen der jeweiligen Fachschaftsräte bleiben in Kraft.
\Abs \Satz Der von der Studentenschaft IHI Zittau gewählte Studentenrat nimmt kommissarisch bis zu den nächsten Wahlen der Studierendenvertretung am IHI Zittau die Rechte und Pflichten der Fachschaft IHI Zittau wahr.


\section{Inkrafttreten}

\Abs \Satz Die Grundordnung und ihre Ergänzungsordnungen treten unmittelbar nach ihrem Beschluss durch den StuRa in Kraft\. Dies gilt für Grundordnungsänderungen entsprechend.

\Abs \Satz Mit dem Inkrafttreten dieser Grundordnung treten alle früheren Satzungen der Studentenschaft der Technischen Universität Dresden außer Kraft.

\end{multicols}

\nopagebreak
\vspace{1cm}
Inkraftgetreten am 04.~Mai~2001.
\\ 
  

\footnotesize
Geändert am 04.~Juli~2003\\
§~18 Abs.~1 : einfügen in Satz zwei von "` , gegebenenfalls unberührt von §~7 Abs.~2 Nr.~2 auch zusätzlich,"\\
§~18 Abs.~4: Satz zwei wird Satz drei; NEU Satz zwei.\\

Geändert am 10.~August~2006\\
§~3 Abs.~2 : NEU Satz zwei\\
§~9 : gestrichen;  NEU Abs. zwei bis vier\\
§~10 : NEU\\
§~15, alt \S~14 : NEU Abs. drei\\
§~19 : NEU\\
§~20 : NEU\\
§~21, alt §~18: Anpassung an geänderten Sitzungsrhythmus

Geändert am 17.~Juli~2008\\
Darlehensordnung ersatzlos gestrichen;\\
Beratungsrichtlinie ersatzlos gestrichen;\\
AE-Ordnung in Finanzordnung integriert;\\
Fachschaftsrahmenordnungen in Grundordnung integriert;\\
In der Satzung, allen Ordnungen, Richtlinien und Durchführungsbestimmungen grammatikalisch maskuline in feminine Formulierungen geändert;\\
Umsortierung der Paragraphen;\\
§~3~Abs.~2~Korrektur der Verweise;\\
§~4~Abs.~1~Korrektur der Verweise;\\
§~5~Abs.~1~Nr.~4 NEU;\\
§~5~Abs.~2~Nr.~3 neu formuliert;\\
alt~§~4~Abs.~2~Nr.~1~und~4 gestrichen;\\
alt~§~4~Abs.~3 gestrichen;\\
§§~6~bis~11 NEU, ehemals Fachschaftsrahmenordnung;\\
§~12~NEU;\\
alt~§~17~Abs.~3 gestrichen;\\
§~14~NEU, ehemals §~18~und~§~39 der Finanzordnung;\\
§~15~Abs.~1, alt~§~7~Abs.~1 "`nach Maßgabe der Fachschaftsrahmenordnung"' gestrichen;\\
§~15~Abs.~6 NEU;\\
alt~§~6~Abs.~2~Nr.~7 gestrichen;\\
§~17~Abs.~3, alt~§~11~Abs.~3 geändert in "`Die Protokolle der StuRa-Sitzungen sind zu veröffentlichen."';\\
§~20~Abs.~3, alt~§~15~Abs.~2 Korrektur der Verweise;\\
§~22~Abs.~1, alt~§~10~Abs.~1 "`des Sitzungsvorstands"' eingefügt;\\
§~23~Abs.~1, alt~§~19~Abs.~1 geändert in "`Der Sitzungsvorstand besteht aus drei vom StuRa gewählten Mitgliedern."';\\
§~23~Abs.~2, alt~§~19~Abs.~2 gestrichen;\\
§~23~Abs.~3, alt~§~20~Abs.~2 Korrektur der Verweise;\\
§~23~Abs.~4, alt~§~20~Abs.~3 "`und Verwaltung"' eingefügt;\\
alt~§~22~Abs.~7 gestrichen;\\
§~23~Abs.~5 NEU;\\
alt~§~20~Abs.~4 gestrichen;\\
§~24,~25,~26,~27 NEU;\\
alt~§§~18,~19,~23,~25,~26,~27 gestrichen;\\
§~28~Abs.~4, alt~§~30~Abs.~2 vollständig neu gefasst;\\
alt~§~31~Abs.~1~Nr.~3~und~6 gestrichen;\\
alt~§~31~Abs.~3~S.~2 gestrichen;\\
§~33~NEU;\\
Geändert am 18.~Dezember~2008\\
In §~6Grundschulpädagogik in Allgemeinbildende Schulen/Grundschule umbenannt;

Geändert am 16.~Juli~2010\\
§~23~Abs.~4 "`Veröffentlichung"' hinzugefügt;\\
§~15~Abs.~1 Satz 2 "`Eine gesonderte Vertretung nach §~75~Abs.~1~S.~7 SächsHG existiert nicht\."' gestrichen;\\
§~15~a hinzugefügt;

Geändert am 13. August 2010\\
§~15~Abs.~4 Satz 3 hinzugefügt;\\
§~20~Abs.~1 "`mit aktivem Stimmrecht"' eingefügt;\\
§~5~a~hinzugefügt;\\
§~9~Abs.~2 "`Der FSR wählt die Vertreterinnen der Gruppe der Studenten in den jeweiligen Fakultätsrat\. Sie müssen Mitglied der Fakultät, nicht jedoch des FSR sein\. Bestehen in einer Fakultät mehrere FSR, so werden die Vertreterinnen in den Fakultätsrat durch den Konvent gewählt."' ersetzt;\\
§~26~Abs.~2 "`für die Dauer ihrer Amtsperiode"' eingefügt;\\
§~15~Abs.~6 eingefügt\\
§~21~Abs.~4 Satz 2 hinzugefügt;\\
§~26~Abs.~3 "`und die Erstellung des vierteljährlichen Rechenschaftsberichtes"' eingefügt;\\
§~4a~hinzugefügt, als Ersatz für § 21 Geschäftsordnung;\\
§~12~Abs.~3 "`gleiches gilt für Mitglieder von Referaten"' hinzugefügt;\\
§~16~Abs.~2~Punkt~4 eingefügt;\\
§~25~Abs.~2 dementsprechend gekürzt und angepasst;\\
§~27~a hinzugefügt;\\
§~14~Abs.~4 hinzugefügt;\\
§~23~a hinzugefügt;\\
§~23~Abs.~1 Satz 2 dementsprechend hinzugefügt;\\
§~24~a~neu;\\
§~4~Abs.~3 dementsprechend "`Ausschuss"' in "`Kommission"' geändert; um hier nicht die Bedingung für Ausschüsse erfüllen zu müssen;\\
§~24~b hinzugefügt;\\
§~20~Abs.~5 hinzugefügt;\\

Geändert am 7. Juli 2011\\
Umbennung der Satzung in Grundordnung: Der Begriff Satzung wird zur Übersichtlichkeit auch in den Übersichten und den Verlauf in Grundordnung geändert. Der Begriff Satzung ist, wenn er auf Ergänzungsordnungen und Dokumenten der Studentenschaft verwendet mit dem der Grundordnung gleichbedeutend.\\

Geändert am 24. Mai 2012\\
§ 6 Abs. 1 Nr. 20 "`Wasserwesen"' durch "`Hydrowissenschaften"' ersetzt; \\

Geändert am 24. Mai 2012\\
§ 28 b NEU; \\

Geändert am 30. August 2012\\
§ 5 Abs. 4 NEU; \\

Geändert am 08. November 2012\\
§ 6 Abs. 1 Nr. 8 "`Grundschulen"' gestrichen; \\

Geändert am 07. Februar 2013\\
§ 6 Abs. 1 Nr. 23 IHI Zittau '"Studierendenschaft IHI"' hinzugefügt; \\
§ 18 Abs. 3 hinzugefügt; \\
§ 18 Abs. 2, Abs. 3 geändert in Ausnahme von Abs. 1 ist [\ldots]; \\
§ 33 Abs. 2 hinzugefügt; \\

Geändert am 25. Oktober 2013 \\
§ 24 Abs. 1 geändert in: "`Ein Ausschuss besteht aus 4 bis 7 Mitgliedern des StuRa, welche zum Zeitpunkt ihrer Wahl über das aktive Stimmrecht im StuRa verfügen. Sie werden vom Studentenrat für die laufende Legislatur der Legislative gewählt."'; \\
§ 24 a Abs. 1 geändert in: "`Der Förderausschuss ist ein ständiger Ausschuss. Er tagt in der Vorlesungszeit wöchentlich, in der vorlesungsfreien Zeit in einem regelmäßigen, zuvor zu veröffentlichendem Rhythmus."'; \\
§ 24 a Abs. 2 geändert in: "`Der Förderausschuss setzt sich aus der Geschäftsführerin Finanzen, sowie vier bis sechs weiteren, gemäß §24 Abs.1 gewählten Mitgliedern zusammen."'; \\
§ 24 a Abs. 5 hinzugefügt; \\
§ 24 a Abs. 6 hinzugefügt; \\
§ 22 Abs. 3 "`14"' durch "`10"' ersetzt; \\

Geändert am 27. November 2014 \\
§ 29 Abs. 1 5. und 6. hinzugefügt; \\

\normalsize
~\\*[4cm]
\begin{center}
\hspace*{\fill}
\parbox{7cm}{Jan-Malte Jacobsen\\GF Hochschulpolitik}
\hfill\parbox{7cm}{Robert Georges\\GF Finanzen}
\hspace*{\fill}
\end{center}     
